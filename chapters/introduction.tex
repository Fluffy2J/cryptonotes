\chapter{Crypto, The Basics}
\section{Cryptography and Modern Cryptography}
    
    Modern cryptography is the scientific study of techniques for securing digital information, transactions, and distributed computations.
    
\section{Principles of Secure Communication}
    There are two types of encryption systems: symmetric (\textbf{private key}, or classical) and asymmetric (\textbf{public key}) encryption. Both rely on these principles for secure communication.\\
    \\
    \textbf{Confidentiality}: An adversary cannot see which messages are transmitted, though they may see that something is being sent.\\
    \\
    \textbf{Authenticity}: An adversary cannot cause non-original messages to be accepted. This is related to but different from...\\
    \\
    \textbf{Integrity}: An adversary cannot alter messages in transit without the alteration being detected.\\

\section{Private Key Encryption}

    Private-key encryption implies both the sender and the receiver use the same secret key to encrypt and respectively decrypt the message.\\
    \\
    The private key encryption scheme is comprised of three algorithms:
    
    \begin{itemize}
      \item The key generation algorithm(\textit{Gen}) which is a probabilistic algorithm that outputs a key, $\ck$, chosen according to some distribution that is determined by the scheme.
      \item The encryption algorithm(\textit{Enc}) which takes as input a key $\ck$ and a plaintext message \textit{M} and outputs a cipher text \textit{C}: $Enc_\ck(M) = \cc;$
      \item The decryption algorithm(\textit{Dec}) which takes as input a key $\ck$ and a cipher \textit{C} and outputs the original message \textit{M}: $Dec_\ck(C) = \cm;$
    \end{itemize}
    
    Decrypting a cipher text using the appropriate key yields the original message or: $Dec_\ck(Enc_\ck(\cm))=\cm$.
    \\
    
% Not sure if this is the best place for this, but it is an important principle, so perhaps...
\section{Kerckhoff's Principle}
    Kerckhoff's principle is fundamental to cryptography. It is this: \textbf{``Security should rely solely on the secrecy of the key.''}.\\
    \\
    Arguments:
    \begin{itemize}
      \item It is much easier to maintain secrecy of a short key than a complex algorithm.
      \item Changing a disclosed key is much easier than changing a disclosed algorithm.
      \item In case more parties need to communicate it is easier to produce multiple keys than multiple algorithms.
    \end{itemize}
    Example of attack scenarios:
    \begin{itemize}
      \item Cipher-only attack
      \item Known-plaintext attack
      \item Chosen-plaintext attack
      \item Chosen-cipher attack
    \end{itemize}

\section{Perfectly Secret Encryption}
  \textbf{Definition:} An encryption scheme $(Gen,Enc,Dec)$ over a message space $\cm$ is perfectly secret if for every probability distribution over $\cm$, every message $\cm \in M$, and every ciphertext $\cc \in C$ for which $P(C=\cc)>0:$

  $$P(M=\cm|C=\cc)=P(M=\cm)$$

  In other words a scheme is perfectly secret if the distributions over messages and ciphertexts are independent.

\section{One Time Pad}
  The one time pad is a good system to review because it is \textbf{totally secure} / perfectly secret, if used correctly. Sadly, it is difficult to use in the real-world.\\
  \\
  It works by XOR-ing the entire message by the cipher. This means that the cipher must be as long as the message, which is clearly a limitation in the real world --- encrypting your hard drive would require another hard drive of data for the cipher.\footnote{The XOR operations outputs a 0 bit at every position in which both operands have the same bit, and a 1 in positions where they differ.}\\
  \\
  The other limitation is the fact it can only be used once. This is because an adversary  can ask you to encrypt a message ($\cc=\cm\oplus k$). They can then use that to calculate the key, $\ck$,  because they have both $\cc$ and $\cm$ ($\ck=\cm\oplus \cc$). In practice, this means that if you know part of a message being sent, you can use that to calculate the bits of cipher in those positions.

\section{Shannon's Theorem}
  
  Shannon's theorem provides a generalised definition for deeming whether a cryptosystem is \textbf{perfectly secret}. First, we assume we have a system with equal numbers of keys, plaintext, and ciphertexts; $\mathbf{\#K = \#P = \#C}$.\\
  \\
  We say that this system provides perfect secrecy iff:
  \begin{itemize}
    \item The probability of any key being used is $1/\#K$.
    \item For each $m \in P$ there exists a $c \in C$.
  \end{itemize}
  It's worth briefly looking into the \textit{why} of Shannon's theorem. If $\#K = \#P = \#C$, then given a key, there must be one mapping of each plaintext to each ciphertext. And if we try every key, then a given message must map to every ciphertext. If it does not, then we have leaked information about the message, for example if $m_1$ maps to $c_{666}$ more than once when encrypted with every key, then given $c_{666}$ we can say of all messages, it is more likely to be $m_1$ than another message, despite not even knowing the key!