\chaptersub{Asymmetric Modes}{Public Key Cryptography}

\section{Introduction}

	The basic concept of public key cryptography is the idea of a box with two keys, a public key, and a private key. The \textbf{public key} allows anyone to \textbf{leave a message}. They cannot however read each others messages. The \textbf{private key} by contrast, allows its owner to \textbf{read any message left}. It is this asymmetry that gives the scheme its name.\\
	\\
	Or to put	it more concretely:\\
	\\
	\textbf{Message + \textcolor{B}{Public Key} = Ciphertext}\\
	\textbf{Ciphertext + \textcolor{R}{Private Key} = Message}\\

	Throughout this chapter, you will see a lot of colours. If you see \textcolor{R}{red}, the highlighted symbols are relevant to the \textcolor{R}{private key}. If you see \textcolor{B}{blue}, the highlighted symbols are relevant to the \textcolor{B}{public key}.

\section{Vanilla RSA}

	RSA is actually very simple. Lets start with the encryption and decryption definitions, and work back from there.

	$$ c = m^{\textcolor{B}{e}} \bmod \textcolor{B}{N} $$

	$$ m = c^{\textcolor{R}{d}} \bmod \textcolor{B}{N} $$
	Simple right? $e$, $d$, and $N$ are the only things we need. So now lets cover just what these are.\\
	\\
  \begin{tabularx}{\linewidth}{l l X}
  \textbf{Symbol} & \textbf{Maths}\footnote{Any unfamiliar symbols or functions are described in the Mathematics section.} & \textbf{Notes}\\
  $\textcolor{B}{N}$ & $\textcolor{B}{N} = \textcolor{R}{p} * \textcolor{R}{q}$ & Where $p$ and $q$ are two massive \textbf{prime} numbers.
  \\
  $\textcolor{B}{e}$ & $\gcd(\textcolor{B}{e},\phi(\textcolor{B}{N}))=1$ & A randomly chosen integer, where $1 < \textcolor{B}{e} < \phi(\textcolor{B}{N})$.
  \\
  $\textcolor{R}{d}$ & $\textcolor{B}{e}*\textcolor{R}{d} = 1 \mod \phi(\textcolor{B}{N})$ & Computed using the XGCD algorithm.
  \\
  \end{tabularx}


\section{Signatures}

\section{Hybrid Encryption}

\section{Key Encapsulation}

\section{Padding Schemes}

\section{Rabin}

\section{ElGamal}