\chapter{Recap}
	% Summary and comparison section
	\section{Block Ciphers}
	Block ciphers cut the message into smaller blocks to be encrypted. The \textbf{Electronic Code Book} encrypts them all individually. \textbf{Cipher Block Chaining} XORs each block of the message with the previous blocks ciphertext. The first message block is XORed with the \textit{Initialisation Vector} and that is sent in plain text as the first block of the ciphertext. \textbf{Counter Mode} generates a pseudo-random bit-string to use in a \textit{One Time Pad}. The IV is again sent in plaintext and is the seed for the cipher-stream. Each block adds one to the counter and encrypts it to get a bit-string to XOR with the message.
	\begin{table}[htp!]
		\centering
		\begin{tabular}{llll}
			\toprule
			Scheme & Mode & \multicolumn{2}{l}{Passes up to}\\
			&& \multicolumn{2}{l}{security model for}\\\cmidrule(r){3-4}
			&& IND & OW\\\midrule
			Block Cipher & Electronic Code Book & --- & CPA\\
			& Chained Block Cipher & CPA & CPA\\
			& Counter & CPA & CPA\\
			\bottomrule
		\end{tabular}
	\end{table}