\chapter{Classical Cryptography}

\section{Historical Ciphers and Their Cryptanalysis}

	\begin{itemize}
	    \item \textbf{Caesar cipher}\\\\
	The encryption is done by rotating the letters of the alphabet by 3(A becomes D). The biggest problem
	with this cipher is that the method is fixed and there is no key to speak of. Therefore anyone
	learning the decryption algorithm would be able to get the initial message effortlessly.
	    \item \textbf{The shift cipher}\\\\
	This cipher is similar to the Caesar cipher but it introduces a secret key \textit{k} that is a number
	between 0 and 25. The encryption is then done by shifting the letters by \textit{k} places.\\\\
	$Enc(\cm_{i}) = [(\cm_i+\ck) \mod 26];\\
	Dec(\cc_{i}) = [(\cc_{i}-\ck) \mod 26]$\\\\
	This cipher is not secure as it can be easily broken by doing an exhaustive search. There are
	only 26 possible keys therefore the cipher does not follow the insufficient key space principle
	which says that any secure encryption scheme must have a key space that is not vulnerable to
	exhaustive search. The number of possible keys must be very large (at least 2\textsuperscript{60} or 2\textsuperscript{70}).
	    \item \textbf{Mono-alphabetic substitution}\\\\
	The idea behind the mono-alphabetic substitution is to map each plaintext character to a
	different cipher text character in an arbitrary manner. The keys space thus consists of all
	permutations of the letters of the alphabet which is approximately 2\textsuperscript{88}, too big for an exhaustive search.\\
	However this does not make it secure. Considering that the plain text message is written
	in plain English we can calculate the frequency of each letter and then compare the cipher`s table of frequencies with the table of frequencies for the English language.
	    \item \textbf{Improved attack on the shift cipher}\\\\
	The initial approach was to try all possible keys and check which one gives a message that
	``makes sense''. But sometimes it is hard to define what ``makes sense'' actually means. If the
	initial message is not written in plain English it is hard to know which solution is the correct one.
	There are cases though when the plain text message is not written in valid English but has the
	letter distribution of such a text.\\
	In this case we compute the following sum: \begin{math}I_j = \sum_{i=0}^{25} p^2*\approx0.065\end{math} where \textit{p\textsubscript{i}} is each letter`s frequency in the plain text with \begin{math}0<=i<=25\end{math}. We know that the letters are shifted by \textit{k} spaces, therefore \textit{q\textsubscript{i+k} = p\textsubscript{i}}. Now all we need to do is compute the sum \begin{math}I_j = \sum_{i=0}^{25} p_i*q_{i+j}\end{math} where \textit{q\textsubscript{i}} is each letter`s frequency in the cipher text and compare it to 0.0065. The \textit{j} for which I\textsubscript{j} is closest to 0.065 is the key. 

	    \item \textbf{The Vigenere cipher}\\\\
	The statistical attack on the mono-alphabetic substitution cipher was possible because the
	mapping of each letter was fixed.\\
	Consider the possibility when a letter is mapped to multiple letters. In this case the table of
	frequencies would be irrelevant as the frequencies are very similar.
	The Viginere cipher takes a repeated small string and forms the key which is then added to the
	original message.\\
	In order to simplify the breaking process we first need to identify the length of the key. If the
	original message is written in plain English then we should look for repeated sequences of length
	2 and 3 in the key. These will indicate us the appearances of common words like \textit{``the''} which
	are in the same relative position of the small substring that is repeated, therefore the distance
	between 2 such appearances would be a multiple of the length of the substring. Knowing this
	we can assume that the greatest common divisor between the distances of such appearances is
	either the length of the key or a multiple of it.\\
	Knowing the key`s length the task is much simplified. We notice that we can split the initial text
	in a number of substrings equal to the keys length(\textit{n}). This way the characters on positions \textit{1,n, 2n} etc. will be encrypted using the first character of the key, the positions \textit{2, n+1, 2n+1} etc. using the second character of the key and so on.\\
	We now have \textit{n} different ciphers that are encrypted using the shift cipher with key \textit{K[i]}.
	Therefore we need to decipher each string individually. Considering that we selected dispersed
	characters from a plain English text, all the substrings will not be in plain English. Knowing this
	we can now apply the method described above under \textit{``Improved attack on the shift cipher''}.
	\end{itemize}

\section{Perfectly Secret Encryption}
	\textbf{Definition:} An encryption scheme $(Gen,Enc,Dec)$ over a message space $\cm$ is perfectly secret if for every probability distribution over $\cm$, every message $\cm \in M$, and every ciphertext $\cc \in C$ for which $P(C=\cc)>0:$

	$$P(M=\cm|C=\cc)=P(M=\cm)$$

	In other words a scheme is perfectly secret if the distributions over messages and ciphertexts are independent.

\section{One Time Pad}
	The one time pad is a good system to review because it is \textbf{totally secure} / perfectly secret, if used correctly. Sadly, it is difficult to use in the real-world.\\
	\\
	It works by XOR-ing the entire message by the cipher. This means that the cipher must be as long as the message, which is clearly a limitation in the real world --- encrypting your hard drive would require another hard drive of data for the cipher.\footnote{The XOR operations outputs a 0 bit at every position in which both operands have the same bit, and a 1 in positions where they differ.}\\
	\\
	The other limitation is the fact it can only be used once. This is because an adversary  can ask you to encrypt a message ($\cc=\cm\oplus k$). They can then use that to calculate the key, $\ck$,  because they have both $\cc$ and $\cm$ ($\ck=\cm\oplus \cc$). In practice, this means that if you know part of a message being sent, you can use that to calculate the bits of cipher in those positions.

\section{Shannon's Theorem}
	
	Shannon's theorem provides a generalised definition for deeming whether a cryptosystem is \textbf{perfectly secret}. First, we assume we have a system with equal numbers of keys, plaintext, and ciphertexts; $\mathbf{\#K = \#P = \#C}$.\\
	\\
	We say that this system provides perfect secrecy iff:
	\begin{itemize}
		\item The probability of any key being used is $1/\#K$.
		\item For each $m \in P$ there exists a $c \in C$.
	\end{itemize}
	It's worth briefly looking into the \textit{why} of Shannon's theorem. If $\#K = \#P = \#C$, then given a key, there must be one mapping of each plaintext to each ciphertext. And if we try every key, then a given message must map to every ciphertext. If it does not, then we have leaked information about the message, for example if $m_1$ maps to $c_{666}$ more than once when encrypted with every key, then given $c_{666}$ we can say of all messages, it is more likely to be $m_1$ than another message, despite not even knowing the key!