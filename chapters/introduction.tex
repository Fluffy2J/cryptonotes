\chapter{Introduction}
\section{Cryptography and Modern Cryptography}
    
    Modern cryptography is the scientific study of techniques for securing digital information, transactions, and distributed computations.
    
\section{Principles of Secure Communication}
    There are two types of encryption systems: symmetric (\textbf{private key}, or classical) and asymmetric (\textbf{public key}) encryption. Both rely on these principles for secure communication.\\
    \\
    \textbf{Confidentiality}: An adversary cannot see which messages are transmitted, though they may see that something is being sent.\\
    \\
    \textbf{Authenticity}: An adversary cannot cause non-original messages to be accepted. This is related to but different from...\\
    \\
    \textbf{Integrity}: An adversary cannot alter messages in transit without the alteration being detected.\\

\section{Private Key Encryption}

    Private-key encryption implies both the sender and the receiver use the same secret key to encrypt and respectively decrypt the message.\\
    \\
    The private key encryption scheme is comprised of three algorithms:
    
    \begin{itemize}
      \item The key generation algorithm(\textit{Gen}) which is a probabilistic algorithm that outputs a key, $\ck$, chosen according to some distribution that is determined by the scheme.
      \item The encryption algorithm(\textit{Enc}) which takes as input a key $\ck$ and a plaintext message \textit{M} and outputs a cipher text \textit{C}: $Enc_\ck(M) = \cc;$
      \item The decryption algorithm(\textit{Dec}) which takes as input a key $\ck$ and a cipher \textit{C} and outputs the original message \textit{M}: $Dec_\ck(C) = \cm;$
    \end{itemize}
    
    Decrypting a cipher text using the appropriate key yields the original message or: $Dec_\ck(Enc_\ck(\cm))=\cm$.
    \\
    
% Not sure if this is the best place for this, but it is an important principle, so perhaps...
\section{Kerckhoff's Principle}
    Kerckhoff's principle is fundamental to cryptography. It is this: \textbf{``Security should rely solely on the secrecy of the key.''}.\\
    \\
    Arguments:
    \begin{itemize}
      \item It is much easier to maintain secrecy of a short key than a complex algorithm.
      \item Changing a disclosed key is much easier than changing a disclosed algorithm.
      \item In case more parties need to communicate it is easier to produce multiple keys than multiple algorithms.
    \end{itemize}
    Example of attack scenarios:
    \begin{itemize}
      \item Cipher-only attack
      \item Known-plaintext attack
      \item Chosen-plaintext attack
      \item Chosen-cipher attack
    \end{itemize}


