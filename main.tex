\documentclass[oneside,a4paper,12pt]{book}

\usepackage[margin=1in]{geometry}
\usepackage{titlesec}
\usepackage{lipsum} 
\usepackage{xcolor} 

\usepackage{tikz}
\usepackage{subcaption}
\usepackage{ifthen}
\usepackage{tabularx}

\usepackage{fancyhdr}
\usepackage{lastpage}

\usepackage{fixltx2e}

\usepackage{amsmath}
\usepackage{amssymb}


\definecolor{R}{RGB}{220,20,60}
\definecolor{B}{RGB}{39,64,139}
\definecolor{G}{RGB}{0,139,69}
\definecolor{orange}{rgb}{1,0.5,0}

\usepackage{hyperref}
\hypersetup{linktocpage}

\begin{document}

\pagestyle{fancy}
\lhead{\footnotesize \parbox{11cm}{Cryptography Notes} }
\lfoot{\footnotesize \parbox{11cm}{Shared}}
\cfoot{}
\rhead{\footnotesize Lecture Notes}
\rfoot{\footnotesize Page \thepage\ of \pageref{LastPage}}
\renewcommand{\headheight}{24pt}
\renewcommand{\footrulewidth}{0.4pt}

\titleformat{\chapter}[display]
  {\normalfont\normalsize\bfseries\LARGE}
  {\chaptertitlename~\thechapter}{1pc}
  {{\color{brown}\titlerule[2pt]}\vspace{1pc}\MakeUppercase}
    \titleformat{name=\chapter,numberless}[display]
  {\normalfont\normalsize\bfseries\LARGE}{}{1pc}
  {\MakeUppercase}
  
  
  
\newcounter{cryptogamearrows}
\newenvironment{cryptogame}[1]{
    \setcounter{cryptogamearrows}{0}
    \begin{center}
        \begin{tikzpicture}
            \node at (-1,0.75) {#1};
}{
            \draw [thick] (0.6,0.1) rectangle (1.9,0.35-0.5*\arabic{cryptogamearrows});
            \node at (1.25,0.2-0.25*\arabic{cryptogamearrows}) {\textbf{A}};
        \end{tikzpicture}
    \end{center}
}
\newcommand{\cgamearrow}[2]{\node [left] at (0,0-0.5*\arabic{cryptogamearrows}) {#1};\draw [thick][#2] (0,0-0.5*\arabic{cryptogamearrows}) -- (0.5,0-0.5*\arabic{cryptogamearrows});\stepcounter{cryptogamearrows}}
\newcommand{\cgameright}[1]{\cgamearrow{#1}{->}}
\newcommand{\cgameleft}[1]{\cgamearrow{#1}{<-}}  

\newcommand{\cbox}[2]{\begin{tikzpicture}\node [rectangle, draw, fill=#1, text centered, rounded corners] {#2};\end{tikzpicture}}
\newcommand{\gbox}[1]{\cbox{blue!20}{#1}}

\newcommand{\boldcbox}[2]{\begin{tikzpicture}\node [rectangle, draw, thick, fill=#1, text centered, rounded corners] {\textbf{#2}};\end{tikzpicture}}

\newboolean{indcca}\newboolean{indcpa}\newboolean{indpass}
\newboolean{owcca}\newboolean{owcpa}\newboolean{owpass}

\newcommand{\clearattackcolours}{
    \setboolean{indcca}{false}\setboolean{indcpa}{false}\setboolean{indpass}{false}
    \setboolean{owcca}{false}\setboolean{owcpa}{false}\setboolean{owpass}{false}
}

\newcommand{\comment}[1]{}
\newcommand{\attackpasscolour}{green}
\newcommand{\ttackfailcolour}{red!80}
\newcommand{\condbox}[3]{\ifthenelse{ \equal{#1}{#3}
                            }{
                                \ifthenelse{\boolean{#1}}{\boldcbox{\attackpasscolour}{#2}}{\bold{\ttackfailcolour}{#2}}
                            }{
                                \ifthenelse{\boolean{#1}}{\cbox{\attackpasscolour}{#2}}{\cbox{\ttackfailcolour}{#2}}
                            }
                        }

\newcommand{\attacktable}[1]{
    \clearattackcolours
    \setboolean{#1}{true}
    \ifthenelse{\boolean{indcca}}{\setboolean{indcpa}{true}}{}
    \ifthenelse{\boolean{indcpa}}{\setboolean{indpass}{true}}{}
    \ifthenelse{\boolean{indcca}}{\setboolean{owcca}{true}}{}
    \ifthenelse{\boolean{indcpa} \OR \boolean{owcca}}{\setboolean{owcpa}{true}}{}
    \ifthenelse{\boolean{indpass} \OR \boolean{owcpa}}{\setboolean{owpass}{true}}{}
    \begin{tabular}{ccccc}
        \condbox{indcca}{IND-CCA}{#1} & $\rightarrow$ & \condbox{indcpa}{IND-CPA}{#1} & $\rightarrow$ & \condbox{indpass}{IND-PASS}{#1}\\
        $\downarrow$ && $\downarrow$ && $\downarrow$ \\
        \condbox{owcca}{OW-CCA}{#1} & $\rightarrow$ & \condbox{owcpa}{OW-CPA}{#1} & $\rightarrow$ & \condbox{owpass}{OW-PASS}{#1}\\
    \end{tabular}
}

\newcommand{\attacktabletwo}[2]{
    \clearattackcolours
    \setboolean{#1}{true}
    \setboolean{#2}{true}
    \ifthenelse{\boolean{indcca}}{\setboolean{indcpa}{true}}{}
    \ifthenelse{\boolean{indcpa}}{\setboolean{indpass}{true}}{}
    \ifthenelse{\boolean{indcca}}{\setboolean{owcca}{true}}{}
    \ifthenelse{\boolean{indcpa} \OR \boolean{owcca}}{\setboolean{owcpa}{true}}{}
    \ifthenelse{\boolean{indpass} \OR \boolean{owcpa}}{\setboolean{owpass}{true}}{}
    \begin{tabular}{ccccc}
        \condbox{indcca}{IND-CCA}{#1} & $\rightarrow$ & \condbox{indcpa}{IND-CPA}{#1} & $\rightarrow$ & \condbox{indpass}{IND-PASS}{#1}\\
        $\downarrow$ && $\downarrow$ && $\downarrow$ \\
        \condbox{owcca}{OW-CCA}{#1} & $\rightarrow$ & \condbox{owcpa}{OW-CPA}{#1} & $\rightarrow$ & \condbox{owpass}{OW-PASS}{#1}\\
    \end{tabular}
}


\titleformat*{\paragraph}{\large\normalfont}

\title{Cryptography A\&B - Lecture Notes}
\author{Shared}
\date{\today}
\maketitle
\tableofcontents
\mainmatter
\part{Cryptography A}

\thispagestyle{fancy}
\lhead{\footnotesize \parbox{11cm}{Cryptography A} }
\lfoot{\footnotesize \parbox{11cm}{Vlad Otrocol}}
\cfoot{}
\rhead{\footnotesize Lecture Notes}
\rfoot{\footnotesize Page \thepage\ of \pageref{LastPage}}
\renewcommand{\headheight}{24pt}
\renewcommand{\footrulewidth}{0.4pt}

\newcommand{\chaptersub}[2]{\chapter[#1]{#1 \\ \large{\textit{#2}}}}

\chapter{Introduction}
\section{Cryptography and Modern Cryptography}
    
    Classical cryptography was, until the 20th century, considered more of an art than a science as it had no real theory to rely upon.\\
    \\
    Modern cryptography, however, is much more formal and encompasses a great deal more than just secret communication. It also covers: message authentication, digital signatures, electronic auctions and elections, digital currency etc.\\
    \\
    It also differs by the places in which it is found; Classical cryptography was mainly used by the military and intelligence organisations, whilst modern cryptography is used in almost every computer system. It is used when accessing a secure website, when authenticating on a multi-user operating system etc.\\
    \\
    Definition: \textbf{Modern Cryptography is the scientific study of techniques for securing digital information, transactions, and distributed computations.}
    
\section{Public \& Private Key Systems}
    Before looking at specific encryption systems, it is worth remembering these basic principles for secure communication:\\
    \\
    \textbf{Confidentiality}: An adversary cannot see which messages are transmitted, though they may see that something is being sent.\\
    \\
    \textbf{Authenticity}: An adversary cannot cause non-original messages to be accepted. This is related to but different from...\\
    \\
    \textbf{Integrity}: An adversary cannot alter messages in transit without the alteration being detected.\\
    \\
    There are 2 types of encryption systems: symmetric (\textbf{private key}, or classical) and asymmetric (\textbf{public key}) encryption.\\

\section{Private Key Encryption}

    Private-key encryption implies both the sender and the receiver use the same secret key to encrypt and respectively decrypt the message.\\
    \\
    This method was useful for the military in the past as the two parties would be able to physically meet and decide upon a key whereas today achieving that is more difficult as the parties meeting would be inconvenient or impossible most times(think about online money transfers or user authentication).\\
    \\
    However, this method is still being applied in some scenarios like disk/file encryption where the same user at different points in time uses the same secret key to both read and write to a file. Private-key encryption is also widely used in conjunction with asymmetric methods.\\
    \\
    The private key encryption scheme is comprised of three algorithms:
    
    \begin{itemize}
      \item The key generation algorithm(\textit{Gen}) which is a probabilistic algorithm that outputs a key, $\ck$, chosen according to some distribution that is determined by the scheme.
      \item The encryption algorithm(\textit{Enc}) which takes as input a key $\ck$ and a plaintext message \textit{M} and outputs a cipher text \textit{C}: $Enc_\ck(M) = \cc;$
      \item The decryption algorithm(\textit{Dec}) which takes as input a key $\ck$ and a cipher \textit{C} and outputs the original message \textit{M}: $Dec_\ck(C) = \cm;$
    \end{itemize}
    
    Decrypting a cipher text using the appropriate key yields the original message or: $Dec_\ck(Enc_\ck(\cm))=\cm$.
    \\
    
% Not sure if this is the best place for this, but it is an important principle, so perhaps...
\section{Kerckhoff's Principle}
    Kerckhoff's principle is fundamental to cryptography. It is this: \textbf{``Security should rely solely on the secrecy of the key.''}.\\
    \\
    Arguments:
    \begin{itemize}
      \item It is much easier to maintain secrecy of a short key than a complex algorithm.
      \item Changing a disclosed key is much easier than changing a disclosed algorithm.
      \item In case more parties need to communicate it is easier to produce multiple keys than multiple algorithms.
    \end{itemize}
    Example of attack scenarios:
    \begin{itemize}
      \item Cipher-only attack
      \item Known-plaintext attack
      \item Chosen-plaintext attack
      \item Chosen-cipher attack
    \end{itemize}




\chapter{Classical Cryptography}

\section{Historical Ciphers and Their Cryptanalysis}

\begin{itemize}
    \item \textbf{Caesar cipher}\\\\
The encryption is done by rotating the letters of the alphabet by 3(A becomes D). The biggest problem
with this cipher is that the method is fixed and there is no key to speak of. Therefore anyone
learning the decryption algorithm would be able to get the initial message effortlessly.
    \item \textbf{The shift cipher}\\\\
This cipher is similar to the Caesar cipher but it introduces a secret key \textit{k} that is a number
between 0 and 25. The encryption is then done by shifting the letters by \textit{k} places.\\\\
$Enc(\cm_{i}) = [(\cm_i+\ck) \mod 26];\\
Dec(\cc_{i}) = [(\cc_{i}-\ck) \mod 26]$\\\\
This cipher is not secure as it can be easily broken by doing an exhaustive search. There are
only 26 possible keys therefore the cipher does not follow the insufficient key space principle
which says that any secure encryption scheme must have a key space that is not vulnerable to
exhaustive search. The number of possible keys must be very large (at least 2\textsuperscript{60} or 2\textsuperscript{70}).
    \item \textbf{Mono-alphabetic substitution}\\\\
The idea behind the mono-alphabetic substitution is to map each plaintext character to a
different cipher text character in an arbitrary manner. The keys space thus consists of all
permutations of the letters of the alphabet which is approximately 2\textsuperscript{88}, too big for an exhaustive search.\\
However this does not make it secure. Considering that the plain text message is written
in plain English we can calculate the frequency of each letter and then compare the cipher`s table of frequencies with the table of frequencies for the English language.
    \item \textbf{Improved attack on the shift cipher}\\\\
The initial approach was to try all possible keys and check which one gives a message that
``makes sense''. But sometimes it is hard to define what ``makes sense'' actually means. If the
initial message is not written in plain English it is hard to know which solution is the correct one.
There are cases though when the plain text message is not written in valid English but has the
letter distribution of such a text.\\
In this case we compute the following sum: \begin{math}I_j = \sum_{i=0}^{25} p^2*\approx0.065\end{math} where \textit{p\textsubscript{i}} is each letter`s frequency in the plain text with \begin{math}0<=i<=25\end{math}. We know that the letters are shifted by \textit{k} spaces, therefore \textit{q\textsubscript{i+k} = p\textsubscript{i}}. Now all we need to do is compute the sum \begin{math}I_j = \sum_{i=0}^{25} p_i*q_{i+j}\end{math} where \textit{q\textsubscript{i}} is each letter`s frequency in the cipher text and compare it to 0.0065. The \textit{j} for which I\textsubscript{j} is closest to 0.065 is the key. 

    \item \textbf{The Vigenere cipher}\\\\
The statistical attack on the mono-alphabetic substitution cipher was possible because the
mapping of each letter was fixed.\\
Consider the possibility when a letter is mapped to multiple letters. In this case the table of
frequencies would be irrelevant as the frequencies are very similar.
The Viginere cipher takes a repeated small string and forms the key which is then added to the
original message.\\
In order to simplify the breaking process we first need to identify the length of the key. If the
original message is written in plain English then we should look for repeated sequences of length
2 and 3 in the key. These will indicate us the appearances of common words like \textit{``the''} which
are in the same relative position of the small substring that is repeated, therefore the distance
between 2 such appearances would be a multiple of the length of the substring. Knowing this
we can assume that the greatest common divisor between the distances of such appearances is
either the length of the key or a multiple of it.\\
Knowing the key`s length the task is much simplified. We notice that we can split the initial text
in a number of substrings equal to the keys length(\textit{n}). This way the characters on positions \textit{1,n, 2n} etc. will be encrypted using the first character of the key, the positions \textit{2, n+1, 2n+1} etc. using the second character of the key and so on.\\
We now have \textit{n} different ciphers that are encrypted using the shift cipher with key \textit{K[i]}.
Therefore we need to decipher each string individually. Considering that we selected dispersed
characters from a plain English text, all the substrings will not be in plain English. Knowing this
we can now apply the method described above under \textit{``Improved attack on the shift cipher''}.
\end{itemize}


\section{Perfectly Secret Encryption}
\textbf{Definition:} An encryption scheme $(Gen,Enc,Dec)$ over a message space $\cm$ is perfectly secret if for every probability distribution over $\cm$, every message $\cm \in M$, and every ciphertext $\cc \in C$ for which $P(C=\cc)>0:$

$$P(M=\cm|C=\cc)=P(M=\cm)$$

In other words a scheme is perfectly secret if the distributions over messages and ciphertexts are independent.

\section{One Time Pad}
The one time pad is a good system to review because it is \textbf{totally secure} / perfectly secret, if used correctly. Sadly, it is difficult to use in the real-world.\\
\\
It works by XOR-ing the entire message by the cipher. This means that the cipher must be as long as the message, which is clearly a limitation in the real world --- encrypting your hard drive would require another hard drive of data for the cipher.\footnote{The XOR operations outputs a 0 bit at every position in which both operands have the same bit, and a 1 in positions where they differ.}\\
\\
The other limitation is the fact it can only be used once. This is because an adversary  can ask you to encrypt a message ($\cc=\cm\oplus k$). They can then use that to calculate the key, $\ck$,  because they have both $\cc$ and $\cm$ ($\ck=\cm\oplus \cc$). In practice, this means that if you know part of a message being sent, you can use that to calculate the bits of cipher in those positions.

\section{Shannon's Theorem}
...

\chapter{Modern Cryptographic Definitions}


\section{The Three Principles of Modern Cryptology}

\begin{enumerate}
    \item Formulate a rigorous and \textbf{precise definition of security}.
    \item When a cryptographic system relies on an \textbf{unproven assumption}, it must be \textbf{precisely stated}. Furthermore, this assumption \textbf{must be minimal}.
    \item Cryptographic constructions must be accompanied by a \textbf{rigorous proof of security} with respect to the definition formulated in principle 1, revolving around the assumptions stated in principle 2.
\end{enumerate}

\subsection{Principle 1: Formulation of Exact Definitions}

\begin{itemize}
    \item \textbf{Why is this important?}
    \begin{itemize}
        \item \textbf{Importance for design:} We need to have a good understanding of our goal so that
we know when we achieve it. Sometimes our cryptographic system doesn't need to
be as efficient as possible, a simpler version being sufficient.
        \item \textbf{Importance for usage:} When we want to choose an existing cryptographic scheme
we need to have some sort of comparison criteria to be able to tell if it suffices our
application.
        \item \textbf{Importance for study:} For researching different types of cryptographic systems
we need to be able to compare them and measure their security level. Without
a rigorous proof of security the only performance attribute we can measure is
efficiency (which is not very accurate).
    \end{itemize}
    \item \textbf{How do we define security?}
    \begin{itemize}
        \item An encryption scheme is secure if no adversary can compute any function of the
plaintext from the cipher text.
        \item An encryption scheme is considered broken if an adversary learns some function of
the plaintext from the cipher text.
        \item The power of the adversary relates to assumptions regarding the actions the
adversary is assumed to be able to take, as well as the adversary's computational
power.
    \end{itemize}
\end{itemize}

\subsection{Principle 2: Reliance on Precise Assumptions}

Most cryptographic constructions cannot be proven secure unconditionally, therefore it relies on some assumptions which must be precisely stated for the following reasons:

\begin{enumerate}
\item \textbf{Validation of the assumption} If the assumption being relied upon is not precisely stated and presented it cannot be studied and potentially refuted.
\item \textbf{Comparison of schemes} If the assumptions used by two schemes are incomparable, then the one based on the better-studied or the simpler assumptions is preferred.
\item \textbf{Facilitation of proofs of security} A mathematical proof that ``the construction is secure if the assumption holds'' cannot be provided without a precise statement of what the assumption is.
\end{enumerate}

\subsection{Principle 3: Rigorous Proofs of Security}

The first two principles lead naturally to this one. Most proofs in modern cryptography use the \textbf{reductionist} approach, that is, \textit{``Given the Assumption A is true, Construction C is secure according to the given definition.''}

\section{Cryptographic Games}

We can capture a notion of security by a picture representing a game played with adversary $A$. The games have one of two goals and the adversary has a range of powers.

\subsection{Indistinguishability (IND-Security)}

Indistinguishability is a measure of security whereby an adversary offers you (the challenger) two messages, and if you were to encrypt only one of them, he / she would be unable to tell \textbf{which one you have encrypted}.\\
\\
Figure~\ref{fig:ind-sec-sym} shows the cryptographic game which captures this for a symmetric cryptographic scheme. $\csubm{0}$ and $\csubm{1}$ are the two messages the adversary gives you and $\ccast$ is the ciphertext you computed from one of the messages. $b'$ is the answer the adversary gives. Note that $|\csubm{0}|=|\csubm{1}|$, since a difference in size would be a dead give away.\\
\begin{figure}[htp!]
    \centering
    \begin{subfigure}[b]{0.4\textwidth}
        \centering
        \begin{cryptogame}{$b\in \{0,1\}$}
            \cgameleft{$\csubm{m},\csubm{m}$}
            \cgameright{$\ccast=\textrm{Enc}_\ck(\csubm{b})$}
            \cgameleft{$b'$}
        \end{cryptogame}
        \caption{Symmetric Key  Case}
        \label{fig:ind-sec-sym}
    \end{subfigure}
    ~
    \begin{subfigure}[b]{0.4\textwidth}
        \centering
        \begin{cryptogame}{$b\in \{0,1\}$}
            \cgameright{$\cpk$}
            \cgameleft{$\csubm{0},\csubm{1}$}
            \cgameright{$\ccast=\textrm{Enc}_{\csk}(\csubm{b})$}
            \cgameleft{$b'$}
        \end{cryptogame}
        \caption{Public Key Case}
        \label{fig:ind-sec-pub}
    \end{subfigure}
    \caption{IND-Security Games}
    \label{fig:ind-sec}
\end{figure}
\\
Figure~\ref{fig:ind-sec-pub} shows the game design for public key encryption. $\cpk$ is the publicly available key, which all adversaries would have access to while trying to crack your message.\\
\\
A cryptographic system fails a game if an adversary can win the game more often than chance (i.e. $> 50\%$ success rate).

\subsection{One-wayness (OW-Security)}

One-wayness is the property that an attacker, with only the ciphertext, \textbf{cannot decrypt the message}. The layout of the games used for this are in Figure~\ref{fig:ow-sec}.

\begin{figure}[htp!]
    \centering
    \begin{subfigure}[b]{0.4\textwidth}
        \centering
        \begin{cryptogame}{$\cm\in \mathbb{P}$}
            \cgameright{$\ccast=\textrm{Enc}_\ck(\csubm{b})$}
            \cgameleft{$m'$}
        \end{cryptogame}
        \caption{Symmetric Key  Case}
        \label{fig:ow-sec-sym}
    \end{subfigure}
    ~
    % TODO: This game is identical to public key IND-Security. Somethings not right here.
    \begin{subfigure}[b]{0.4\textwidth}
        \centering
        \begin{cryptogame}{$b\in \{0,1\}$}
            \cgameright{$\cpk$}
            \cgameleft{$\csubm{0},\csubm{1}$}
            \cgameright{$\ccast=\textrm{Enc}_{\csk}(\csubm{b})$}
            \cgameleft{$b'$}
        \end{cryptogame}
        \caption{Public Key Case}
        \label{fig:ow-sec-pub}
    \end{subfigure}
    \caption{OW-Security Games}
    \label{fig:ow-sec}
\end{figure}

\subsection{Adversarial Powers}
These attacks also have defined `adversarial powers', where the adversary has access to specific oracles.
\begin{itemize}
    \item \textbf{Passive Attack}: Has no oracles --- all games above are passive attacks.
    \item \textbf{Chosen Plaintext Attack (CPA)}: The adversary can encrypt any mesage of his/her choosing.
    \item \textbf{Chosen Ciphertext Attack (CCA)}: The adversary can decrypt any message of his choosing, except he is not allowed to decrypt $\ccast$.
\end{itemize}
We assume that if an adversary has access to a decryption oracle, then they have access to the encryption one, so CCA is an extension of CPA\footnote{Except in one case we will see later on concerning hybrid encryption schemes.}. There is no notion of a passive attack for public key encryption because the public key (which is used to encrypt) is public, so the adversary always has an encryption oracle.

\section{Reductions}
We can make comparisons of problems by reducing one to another, thereby defining one as `no harder' than the other. A reduction is where we can, in polynomial time, convert a problem into another one. We say \textbf{Problem A is no harder than Problem B} if we can convert Problem A into Problem B. This is written as \boldmath $A \leq_P B$ \unboldmath.\\
\begin{figure}[htp!]
    \begin{center}
        \begin{tabular}{ccccc}
            \gbox{IND-CCA} & $\rightarrow$ & \gbox{IND-CPA} & $\rightarrow$ & \gbox{IND-PASS}\\
            $\downarrow$ && $\downarrow$ && $\downarrow$ \\
            \gbox{OW-CCA} & $\rightarrow$ & \gbox{OW-CPA} & $\rightarrow$ & \gbox{OW-PASS}\\
        \end{tabular}
    \end{center}
    \caption{Relationships between attacks}
    \label{fig:relations}
\end{figure}
\\
Some attacks are more powerful than others. Figure~\ref{fig:relations} show these relationships between the attacks. An arrow ($A \rightarrow B$) means $A$ is more powerful than $B$ and thus a proof that a system meets $A$'s notion of security, also proves it meets $B$'s.\\
\\
\textbf{IND-CCA is the de-facto} security definition we should accept. This means that the encryption must be probabilistic (i.e. encryption is a one-to-many function) because it could ask the oracle to encrypt $m_0$ and from that could win the game.


\section{Stream Ciphers}
The idea behind a stream cipher is to replace the (possibly) huge cipher needed for the Vernam scheme by a pseudo-random sequence which is `seeded` by a key of a more practical size. There's some confusion over whether the term `stream cipher' refers to the algorithm generating the stream or the entire encryption scheme. The term can use both, but the book accompanying the unit recommends you only use it for the algorithm
\begin{figure}
    %\includegraphic[width=\textwidth]{img/streamchipher.png}
    \caption{A stream cipher using a DFA where keys determine state update ($\csubk{1}$), initial state ($\csubk{2}$) and output filter ($\csubk{3}$)}
\end{figure}
\\
One model of a stream cipher is of a finite state machine, where the key provides the initial state, variables used to convert a keystream, $\csubk{i}$, to the next keystream, $\csubk{i+1}$. The ouput of the FSA is also XORed with a value taken from the key.\\
\\
One source for confusion may come from that fact that the slides use $\csubk{x}$ to for both the key values which are inputted into the stream cipher and for the keystream, which is the output of the stream cipher.\\
\\
Decryption is easy as both sender and receiver use key to generate the keystream of necessary size. This possible because the algorithm is deterministic, which, as meantioned later, is a source of weakness.


\chaptersub{Symmetric Modes}{Private Key Cryptography}

\section{Block Cipher Modes}\label{sec:blockciphermodes}
    A \textbf{block cipher} is type of encryption based on \textbf{permutation}, where they key, $\{0,1\}^k$, defines a transposition of bits in the block to the encrypted ciphertext block, $\{0,1\}^b$. The block is a proportion of the message.\\
    \\
    DES was the first civilian block cipher and was developed at IBM in the 1970s. When the US government adopted it, they recommended the following four ways to use it; these are now used with any block cipher.
    \begin{itemize}
        \item Electronic Code Book
        \item Cipher Block Chaining
        \item Counter
        \item ...
    \end{itemize}
    
    \subsection{Electronic Code Book}
    This is a very simple, it \textbf{divides} the message into blocks of size $b$, pads the last one, and \textbf{encrypts them all individually}. This is ineffective to say the least, as evidenced in Figure~\ref{fig:tux}. A block cipher using ECB is only OW-CPA, as Figure~\ref{fig:ecb-attacktable} says. 
    This is terrible!\\
    \begin{figure}[htp!]
    \centering
    \attacktable{owcpa}
    \caption{Security Models ECB passes}
    \label{fig:ecb-attacktable}
    \end{figure}
    \\
    The weakness of this mode lays in the fact that the blocks are encrypted independently. This is the basis for the two attacks below, but it also means it's susceptible to \textbf{block replay}. That is where an adversary edits a bit knowing how it will affect the decrypted message. If I knew you were sending me some money, and I knew the block containing the amount you were transferring, I could change that block in the hope I would end up with a larger transaction. This could be fixed with a checksum. ECB has one positive aspect though; an error in one block of the ciphertext will not propagate to other blocks during decryption or encryption.\\
    \\
    \textbf{Proving a cryptographic system passes a security model is beyond the scope of this course}, but you should be able to give an intuitive reason. It passes OW-CPA because, with only an encryption oracle, you would have to brute-force every possible message to match it with the ciphertext. We can always make $b$ large enough so that this is not possible\footnote{This might be possible if you knew the context of the message, if you understand what is being sent and the domain of possible values of a block is relatively small. This is simply poor implementation of the encryption, however, and we don't worry about that.}. Note that all the proofs for block cipher modes assume we are using a `perfect block cipher'.\\
    \\
    \textbf{OW-CCA Example:} We can win a OW-CCA game by kind of cheating in the following way. If we have a decryption oracle, we can decrypt anything that is not the message. Since the blocks are independent, we can simply split the message in half and decrypt both individually and then concatenate the result.\\
    \\
    \textbf{IND-PASS Example:} In an indistinguishability game we decide the two messages sent to the oracle (of which one will be encrypted). Again we use the fact that the blocks are encrypted independently, and give one of the messages as a bit string concatenated onto itself, $m_0=b_1||b_1$. If the ciphertext can be split into two equal bit strings in the same way, then $m_0$ was encrypted, else $m_1$.
    
    \subsection{Cipher Block Chaining}
    Cipher Block Chaining removes a lot of the problems found in ECB by XORing each block with the previously encrypted block --- incorporating a dependence on the previous block. The first block of the ciphertext is called the \textbf{Initalisation Vector} (IV), and is what the first block of the message is XOR-ed with.\\
    \begin{figure}[htp!]
        \centering
        \includegraphics[width=10cm]{img/cbc}
        \caption{Diagram of CBC encrypting}
    \end{figure}
    \\
    More formally:\nopagebreak
    \begin{center}
    \begin{tabular}{lll}
    \textbf{Encryption}:                                        && \textbf{Decryption}\\
    $\csubc{0} = IV$                                                  && $IV = \csubc{0}$\\
    $\csubc{1} = Enc_\ck(\csubm{1} \oplus IV)$                                && $\csubm{1} = Dec_\ck(c_1) \oplus IV$\\
    $\csubc{i} = Enc_\ck(\csubm{i} \oplus \csubc{i-1}) \textrm{ for } i > 1$      && $\csubm{i} = Dec_\ck(c_i) \oplus \csubc{i-1} \textrm{ for } i > 1$\\
    \end{tabular}
    \end{center}
    \begin{figure}[htp!]
        \centering
        \attacktable{indcpa}
        \caption{Security of Models CBC}
        \label{fig:cbc-attacktable}
    \end{figure}
    Note that $IV$ is sent unencrypted, because it is needed for decryption. This can seem pointless, but if it is generated randomly every time, then it means that encryption is probabilistic. CBC is OW-CPA and IND-CPA but not OW-CCA or IND-CCA.\\
    \begin{figure}[htp!]
        \centering
        \begin{subfigure}[b]{0.3\textwidth}
            \centering
            \includegraphics[width=\textwidth]{img/Tux.jpg}
            \caption{Original Image}
        \end{subfigure}
        \begin{subfigure}[b]{0.3\textwidth}
            \centering
            \includegraphics[width=\textwidth]{img/Tux_ecb.jpg}
            \caption{Encrypted using ECB}
        \end{subfigure}
        \begin{subfigure}[b]{0.3\textwidth}
            \centering
            \includegraphics[width=\textwidth]{img/Tux_secure.jpg}
            \caption{Encrypted using CBC}
        \end{subfigure}
        \caption{Bitmap image of Tux being encrypted using ECB and CBC.}
        \label{fig:tux}
    \end{figure}
    \\
    \textbf{With a decryption oracle, CBC will fail} because of the same trick used before --- we can ask to oracle to decrypt the ciphertext with extra blocks on the end.
    
    
    
    \subsection{Counter Mode}
    Counter mode basically uses a stream cipher to generate a pseudo-random bit-string which can be one-time-padded with message. As Figure~\ref{fig:ctr} shows, the algorithm generates a $IV$ ($IV \leftarrow \{0,1\}^n$) and increments this by one for every block. The $IV+n$ values are then put through the block cipher to generate the bit-string. In all the material, the $IV$ is refereed to as $ctr$ for this algorithm.\\
    \begin{figure}[htp!]
        \centering
        \includegraphics[width=0.5\textwidth]{img/ctr.png}
        \caption{Diagram of CTR mode}
        \label{fig:ctr}
    \end{figure}
    \\
    Formally:
    \begin{center}
        \begin{tabular}{lll}
            \textbf{Encryption:} && \textbf{Decryption:}\\
            $\csubc{0} = IV$ && $IV = \csubc{0}$\\
            $\csubc{i} = Enc_\ck(IV + i) \oplus \csubm{i}$ && $m_i = Enc_\ck(IV + i) \oplus \csubc{i}$
        \end{tabular}
    \end{center}
    CTR mode passes the same security modes as CBC (Figure~\ref{fig:ctr-attacktable}). But it has one positive over CBC: decrypting a block is not dependent on the decryption of the previous block, meaning that we can parallelise decryption (and encryption for that matter).
    \begin{figure}[htp!]
        \centering
        \attacktable{indcpa}
        \caption{Security of Models CTR}
        \label{fig:ctr-attacktable}
    \end{figure}





\section{S-Box \& Analysis}\label{sec:sbox}
    Until now, we haven't talked about what happens in the $Enc$ function. In there, there is a mixture of \textbf{substition} of characters for other characters\footnote{Think of Caesar's cipher where a letter is moved $n$ places in the alphabet, where $n$ is defined in the key} and \textbf{permutations}, where bits (or larger groupings of bits) change position. When there is this mixture of subtition and permutation, we give it the imaginative name of a `substitution-permutation network'.\\
    \\
    One method an attacker might use is called a \textbf{known-plaintext attack}, where they have the \textit{plaintext} and \textit{ciphertext} and try to work the key out from these. This means we need a large number of keys, so that an exhaustive keysearch is not possible. We also have to make sure that the block size is large enough, since we could store known block decryptions and decipher a large part of the ciphertext through this (\textbf{text dictionary attack}).\\
    \begin{table}[htp!]
        \centering
        \begin{tabular}{lccccccccccccccccccccccccccc}
            \toprule
            Input & 0 & 1 & 2 & 3 & 4 & 5 & 6 & 7 & 8 & 9 & A & B & C & D  & E & F\\
            \midrule
            Output & E & 4 & D & 1 & 2 & F & B & 8 & 3 & A & 6 & C & 5 & 9 & E & 7\\
            \bottomrule
        \end{tabular}
        \caption{Example of a look-up table for an S-Box}
        \label{fig:lookupsbox}
    \end{table}
    \\
    An \textbf{S-Box} is a mapping of bit-strings to other bit-strings, theses do not have to be the same length, but if they are not, the output is almost always of a shorter length. Table~\ref{fig:lookupsbox} shows an example of an S-Box which maps 4-bit strings to 4-bit strings. We want to make the substitution as \textbf{non-linear} as possible. Formally, it would be linear if Equation~\ref{equ:sboxlin} was true, or at least was true was large probability.
    % FIXME: Line above ends in confusing way.
    
    \subsection{Differential Analysis}
    At the heart of a number of the techniques used to successfully beat crypto games in Section~\ref{sec:blockciphermodes}, was that we get around the constraint of not being allowed to send the message or ciphertext to an oracle by changing it into something we know will still give us something useful. With a linear S-Box, we (as an attacker) would be able to transform the message/ciphertext and then perform another transformation on the oracles response.
    \begin{equation}
        \begin{array}{l}
            \csubm{1} \rightarrow \csubc{1}\\
            \csubm{2} \rightarrow \csubc{2}
        \end{array}\Bigr\}
        \Rightarrow
        (\csubm{1} + \csubm{2}) \rightarrow (\csubc{1} + \csubc{2})
        \label{equ:sboxlin}
    \end{equation}
    To find patterns that we can exploit to do this, we perform \textbf{differential analysis} on the S-Box. Differential analysis \textit{tabulate[s] specific differences in the input that lead to specific differences in the output with probability greater than would be expected for a random permutation}. This would tell us whether adding 5 to the input will always give us an output 10 too large.\\
    \\
    To do this, we make a table with all the inputs against all the outputs. Then we take \textbf{every pair of inputs}, $(s_1,s_2)$, calculate there difference, $\Delta s=s_1\oplus s_2$, and the the difference between the outputs of those inputs, $\Delta t = \mathbb{S}(s_1) \oplus \mathbb{S}(s_2)$. Figure~\ref{fig:diffanalpair} gives an example of this. For each pair, we add one to the value at $(\Delta s,\Delta t)$, so that cell $(s,t)$ gives us the number of times a difference of $s$ in the input caused a difference in $t$ in the output.
    \begin{figure}[htp!]
        \begin{align*}
            s_1=3=0011_2   && t_1=\mathbb{S}(s_1) = 1 &= 0001_2  &&  \Delta s=s_1\oplus s_2=5=0101_2\\
            s_2=6=0110_2   && t_2=\mathbb{S}(s_2) = B &= 1011_2  &&  \Delta t=t_1\oplus t_2=A=1010_2
        \end{align*}
        \caption{Example of differential analysis for the pair of inputs $(3,6)$}
        \label{fig:diffanalpair}
    \end{figure}
    \\
    Although its unlikely that the table will show all differences in the input of $s$ lead to an output difference of $t$, even a \textit{better than chance} difference gives a cryptanalyst a much smaller space for them them brute force. A perfect S-Box would have a 1 in every cell, but this is \textit{impossible}. We can measure an S-Boxes quality by how close to a \textbf{uniform distribution} it has in this table.
    \subsection{Linear Analysis}
        Linear analysis consists of two parts. The first is to \textbf{construct linear equations} describing relationships between the \textit{plaintext}, \textit{ciphertext} and \textit{key bits} that have a high bias; that is, whose probability of being true is close to 0 or 1.  The second is to use these linear equations in conjunction with known plaintext-ciphertext pairs to derive key bits.\\
        \\
        The notes write a linear equation for an S-Box has n input bits $X_i$ and m-output bits $Y_i$, as

        \begin{equation}
            L_{I,J} = (X_{i_1} \oplus \cdots \oplus X_{i_2}) \oplus (Y_{i_1} \oplus \cdots \oplus Y_{i_2})
        \end{equation}
        \begin{align*}
            \text{where, } & I = \{i_1, \cdots ,i_u\} \subset \{1, \cdots ,n\}\\
            & J = \{j_1, \cdots ,j_u\} \subset \{1, \cdots ,n\}
        \end{align*}
        because they apparently don't want you to understand it. It just means that $L$ is a function taking bits from specific positions from the input and output and XORing them all. Eg, $L_1 = X_1 \oplus X_3 \oplus X_4 \oplus Y_2$.\\
        \\
        For each linear equation, $L_i$, we calculate the probability that for a random input, $X$, the function will return 0: $p=Pr[L_i=0]$. To obtain the bias, we just subtract $-0.5$ from this probability, making the bias in a range of -0.5 to 0.5.\\
        \\
        If function has a large enough bias (either positive or negative), then we can approximate the whole cipher as a linear function. Using this method we can prove that an S-Box is non-linear.

\section{AES --- Rijndael}
    Some ciphers involve repeating a weaker `round function' to make a stronger encryption. Round functions will output bit-strings the same size as the input, and they must be a revertible one-to-one function, so that we can decrypt the ciphertext. The notation usually uses $r$ for the number of rounds, $n$ for the block size and $s$ for the key size. Each round uses a different key, and these sub-keys are all derived from the main key. A block is usually represented as a matrix holding all the bytes. Each state of the encryption is applied to the `state matrix', $s$.\\
    \\
    AES is an encryption scheme which uses round functions. It works on 128-bit blocks, uses keys of 128/192/256-bits and works over 10/12/14 rounds. Each round consists of the following operations:\\
    \newcommand{\leftminipagewidth}{0.3\textwidth}
    \newcommand{\rightminipagewidth}{0.7\textwidth}
    \begin{tabular}{ll}
        \begin{minipage}[t]{\leftminipagewidth}
            \textbf{Byte Substitution}\\
            Just apply the S-Box to the block.
        \end{minipage}
        &
        \begin{minipage}[t]{\rightminipagewidth}
        $
            \begin{pmatrix}
                s_{0,0} & s_{0,1} & s_{0,2}\\
                s_{1,0} & s_{1,1} & s_{1,2}\\
                s_{2,0} & s_{2,1} & s_{2,2}
            \end{pmatrix}
            \rightarrow
            \begin{pmatrix}
                \mathbb{S}(s_{0,0}) & \mathbb{S}(s_{0,1}) & \mathbb{S}(s_{0,2})\\
                \mathbb{S}(s_{1,0}) & \mathbb{S}(s_{1,1}) & \mathbb{S}(s_{1,2})\\
                \mathbb{S}(s_{2,0}) & \mathbb{S}(s_{2,1}) & \mathbb{S}(s_{2,2})
            \end{pmatrix}
        $
        \end{minipage}

        \\\\

        \begin{minipage}[t]{\leftminipagewidth}
            \textbf{Shift Rows}
            Shift a row, so that message is diffused over the columns.
        \end{minipage}
        &
        \begin{minipage}[t]{\rightminipagewidth}
        $
            \begin{pmatrix}
                \textcolor{red}{s_{0,0}} & \textcolor{green}{s_{0,1}} & \textcolor{blue}{s_{0,2}}\\
                \textcolor{red}{s_{1,0}} & \textcolor{green}{s_{1,1}} & \textcolor{blue}{s_{1,2}}\\
                \textcolor{red}{s_{2,0}} & \textcolor{green}{s_{2,1}} & \textcolor{blue}{s_{2,2}}
            \end{pmatrix}
            \rightarrow
            \begin{pmatrix}
                \textcolor{red}{s_{0,0}} & \textcolor{green}{s_{0,1}} & \textcolor{blue}{s_{0,2}}\\
                \textcolor{green}{s_{1,1}} & \textcolor{blue}{s_{1,2}} & \textcolor{red}{s_{1,0}}\\
                \textcolor{red}{s_{2,0}} & \textcolor{green}{s_{2,1}} & \textcolor{blue}{s_{2,2}}
            \end{pmatrix}
        $
        \end{minipage}

        \\\\

        \begin{minipage}[t]{\leftminipagewidth}
            \textbf{Mix Column}\\
            Can choose a matrix to multiply by. Diffuses over rows. \footnotemark
        \end{minipage}
        &
        \begin{minipage}[t]{\rightminipagewidth}
        $
            \begin{pmatrix}
                s_{0,0} & s_{0,1} & s_{0,2}\\
                s_{1,0} & s_{1,1} & s_{1,2}\\
                s_{2,0} & s_{2,1} & s_{2,2}
            \end{pmatrix}
            \rightarrow
            \begin{pmatrix}
                2 & 3 & 1\\
                1 & 1 & 4\\
                3 & 1 & 1
            \end{pmatrix}
            \begin{pmatrix}
                s_{0,0} & s_{0,1} & s_{0,2}\\
                s_{1,0} & s_{1,1} & s_{1,2}\\
                s_{2,0} & s_{2,1} & s_{2,2}
            \end{pmatrix}
        $
        \end{minipage}

        \\\\

        \begin{minipage}[t]{\leftminipagewidth}
            \textbf{Round Key Addition}\\
            XOR with the round key.
        \end{minipage}
        &
        \begin{minipage}[t]{\rightminipagewidth}
        $
            \begin{pmatrix}
                s_{0,0} & s_{0,1} & s_{0,2}\\
                s_{1,0} & s_{1,1} & s_{1,2}\\
                s_{2,0} & s_{2,1} & s_{2,2}
            \end{pmatrix}
            \rightarrow
            \begin{pmatrix}
                k_{0,0} & k_{0,1} & k_{0,2}\\
                k_{1,0} & k_{1,1} & k_{1,2}\\
                k_{2,0} & k_{2,1} & k_{2,2}
            \end{pmatrix}
            \oplus
            \begin{pmatrix}
                s_{0,0} & s_{0,1} & s_{0,2}\\
                s_{1,0} & s_{1,1} & s_{1,2}\\
                s_{2,0} & s_{2,1} & s_{2,2}
            \end{pmatrix}
        $
        \end{minipage}

    \end{tabular}\\\\
    \footnotetext{The aim is to diffuse the message as much as possible, so a maximal distance algorithm is used to find the matrix which will do this the most --- you don't need to worry about that though.}\\
    It's possible that you need to know they are put together in AES, so you can see the pseudo-code in Algorithm~\ref{alg:aes}. Basically, every round the sub-key is added, then the S-Box is applied, then the rows are shifted, then the columns are mixed. In the last round the columns aren't mixed though.\\
    \begin{figure}[htp!]
        \centering
        \begin{algorithm}[H]
            \SetAlgoLined
            AddRoundKey($S$,$K_0$]);\\
            \For{$i\leftarrow 1$ \KwTo $9$}{
                SubBytes($S$);\\
                ShiftRows($S$);\\
                MixColumns($S$);\\
                AddRoundKey($S$,$K_i$);\\
            }
            SubBytes($S$);\\
            ShiftRows($S$);\\
            \caption{AES}
            \label{alg:aes}
        \end{algorithm}
    \end{figure}
    \\
    Clearly we can't just use any S-Box, they have to not be susceptible to differential and linear analysis. The S-Box used is known as the Rijndael S-Box. It can be broken down into two sections:
    \begin{enumerate}
        \item[a)] \textbf{Multiplicative Inverse}: Remember the multiplicative inverse, $x^{-1}$, of $x$ is the number in the finite space for which $x\cdot x^{-1}=1$. So in this context, we want $x\cdot x^{-1} \mod 2^8 = 1$. It must equal 1 because 1 is the identity number for multiplication ($x\cdot I=x$). We count zeros inverse as zero.
        \item[b)] \textbf{Linear map} over $\mathbb{F}_2$, which just means its XORed with a bit-string.
    \end{enumerate}
    Put them together, and you make the equation of the form $\mathbb{S}(x)=A^{-1}+c$. And it turns out, because of the modular arithmetic\footnote{I think...}, it can actually be placed in the form $\mathbb{S}(x) = A'x\oplus c'$. The actual formula is below, where $y_n$ is the $n^\textrm{th}$-bit in the output of the S-Box.\\
    \begin{equation}
        \begin{pmatrix}
            y0\\y1\\y2\\y3\\y4\\y5\\y6\\y7
        \end{pmatrix}
        =
        \begin{pmatrix}
            1&0&0&0&1&1&1&1\\
            1&1&0&0&0&1&1&1\\
            1&1&1&0&0&0&1&1\\
            1&1&1&1&0&0&0&1\\
            1&1&1&1&1&0&0&0\\
            0&1&1&1&1&1&0&0\\
            0&0&1&1&1&1&1&1\\
            0&0&0&1&1&1&1&1
        \end{pmatrix}
        \cdot
        \begin{pmatrix}
            x0\\x1\\x2\\x3\\x4\\x5\\x6\\x7
        \end{pmatrix}
        \oplus
        \begin{pmatrix}
            1\\1\\0\\0\\0\\1\\1\\0
        \end{pmatrix}
    \end{equation}\\
    \textbf{Rijndael's Key Schedule}\\
    The key schedule is the mechanism used to `expand' the 128-bit key into 10 128-bit keys\footnote{The number of rounds is linked t the key length, it is only 10 rounds for 128-bit keys.}. The first round's key is simply the original key. For the rest, the process is as follows (for the $i$'th round):
    \begin{enumerate}
        \item Copy the $i-1$'th round's output into a temporary variable $t$.
        \item Bitwise rotate $t$ left by 1 byte.
        \item Apply the S-Box to each byte of $t$.
        \item XOR the \textbf{first byte only}, with the output of $rcon[i]$.
    \end{enumerate}
    So, what's Rcon I hear you cry? Never fear, I shall explain, though Martijn's slide gloss over this so if you're struggling to remember things as it is \textbf{just remember the above}. The Rcon function is simply defined as the following operation in the Rijndael finite field:
    $$ rcon(i) = x^{i-1} $$
    So this is in effect, another S-Box! Just as with the AES S-Box, we simply use it like a lookup table, hence the square bracket notation in the algorithm.
    \subsection{Why?}
    
\section{Message Authentication}
    \subsection{Why Do This}
        Encrypting data provides confidentiality (remember the three goals), but does not provide authenticity or integrity without additional sauce. This is the additional sauce that provides \textbf{integrity}.
        It comes with two flavours: \textbf{MDC} (Manipulation Detection Codes) and \textbf{MAC} (Message Authentication Codes). We will mainly worry about \textbf{MAC} stuff.
        The reason we use these at all is that they are the secret ingredient in getting from IND-CPA to the mythical, coveted IND-CCA level of security.


    \subsection{\textbf{M}essage \textbf{A}uthentication \textbf{C}odes, you say?}
    MAC codes are the result of hashing the message, using a hash function that takes a key.
    \begin{figure}[htp!]
        \centering
        MAC = $h_{\ck}(\cm)$
    \end{figure}
    You would then typically send the MAC concatenated at the end of the message. The hash function, as per Kerckhoff's principle, is publicly described. Given $\ck$ and $\cm$, computing $h_{\ck}(\cm)$ should be easy.
    Given only $\cm$, computing a correct hash should be very difficult, even if several message-to-hash pairs are already known.\\
    \\
    A hash function is simply a surjective mapping from arbitrarily long strings to fixed length hash strings.

    \subsection{Security Model}
    Our security models are simple:
    \begin{itemize}
        \item \textbf{EF-PASS}: The passive attack where the bad guy can generate a valid MAC for any message, gibberish or otherwise. 
        \textbf{Existential} forgery is where the message may just be random rubbish, \textbf{Selective} forgery is creating a MAC for a specific message.

        \item \textbf{EF-CMA}: As above, but the douche of an adversary has an oracle that can perform MAC generation on other messages of his choice (but not the target message). This is the \textbf{Chosen Message Attack}
    \end{itemize}
    We will now look at two games for Existential Forgery; as a MAC may be probabilistic, we define two algorithms for these games:

    \begin{itemize}
        \item $\textbf{MAC}_{\ck}(\cm)$: Generates the MAC for message \emph{\cm}.
        \item $V_{\ck}(\cm, \cc)$: A boolean-returning verification function. True is good. False is bad. Get with the program, kids. Also this may not simply be a case of recomputing the MAC; if the algorithm that generates it is probabilistic, we need other methods of checking correctness.
    \end{itemize}
    It is a given that our opponent has access to a Verification Oracle (otherwise how will they know if they have cracked it?). 

    \begin{figure}[htp!]
    \centering
    \begin{subfigure}[b]{0.4\textwidth}
        \centering
        \begin{cryptogame}{}
            \cgameright{$V(\cm, \cc)$}
            \cgameleft{$\cmast$, $\ccast$}
        \end{cryptogame}
        \caption{EF-PASS: Passive Attack}
        \label{fig:ef-pass}
    \end{subfigure}
    ~
    \begin{subfigure}[b]{0.4\textwidth}
        \centering
        \begin{cryptogame}{}
            \cgameright{$V(\cm, \cc)$}
            \cgameright{$c = MAC_{\ck}(\cm)$}
            \cgameleft{$\cmast$, $\ccast$}
        \end{cryptogame}
        \caption{EF-CMA: Chosen Message Attack}
        \label{fig:ef-cma}
    \end{subfigure}
    \caption{MAC security games}
    \label{fig:ef-games}
\end{figure}

    There exists a `Strong Forgery' variant of \textbf{EF-CMA} called, unsuprisingly, \textbf{SF-CMA}, which changes the restriction on the MAC oracle to be that whilst $m^{*}$ can be passed to the oracle, $c^{*}$ must not have been returned.

    Assuming we can create a MAC function that achieves this, we can now make INC-CCA symmetric encryption schemes! Hooray!

    There exists a `Strong Forgery' variant of \textbf{EF-CMA} called, unsuprisingly, \textbf{SF-CMA}, which changes the restriction on the MAC oracle to be that whilst $\cmast$ can be passed to the oracle, $\ccast$ must not have been returned.\\
    \\
    Assuming we can create a MAC function that achieves this, we can now make INC-CCA symmetric encryption schemes! Hooray!

    \subsection{IND-CCA Here We Come}
    The reason that CBC and CTR modes, whilst groovy, were not \textbf{IND-CCA} was that an attacker could look at our ciphertext and construct a related one which could be decrypted through their oracle, giving them a related plaintext.

    MAC prevents them from doing that! Sweet! So to make an \textbf{IND-CCA} secure scheme you will need:
    \begin{itemize}
        \item One \textbf{IND-CPA} secure symmetric cipher \emph{E}.
        \item One \emph{SECURE}\footnote{i.e. Cannot produce a valid \textbf{MAC} without access to the correct key} \textbf{MAC} function \emph{MAC}.
        \item One hybrid key consisting of $\csubk{0}$ and $\csubk{1}$ which are the keys for \emph{E} and \emph{MAC} respectively.
    \end{itemize}

    We can then construct encryption and decryption thusly:
    \begin{figure}[htp!]
    \centering
    \raisebox{-0.5\height}{
        \begin{subfigure}[b]{0.4\textwidth}
            \centering
            \textbf{Encrypt}
            \begin{enumerate}
                \item Split $\ck$ into $\csubk{0}$ and $\csubk{1}$
                \item $\csubc{0} = E_{\csubk{0}}(\cm)$
                \item $\csubc{1} = MAC_{\csubk{1}}(\csubc{0})$
                \item And voila: $\cc = (\csubc{0}, \csubc{1})$
            \end{enumerate}
            \label{fig:ind-cca-enc}
        \end{subfigure}
        }
    ~
    \raisebox{-0.5\height}{
        \begin{subfigure}[b]{0.4\textwidth}
            \centering
            \textbf{Decrypt}
            \begin{enumerate}
                \item Split $\ck$ into $\csubk{0}$ and $\csubk{1}$
                \item Split $\cc$ into $\csubc{0}$ and $\csubc{1}$
                \item $\csubcast{1} = MAC_{\csubk{1}}(\csubc{0})$
                \item If $\csubcast{1} \neq \csubc{1}$ then ABORT and return $\bot$
                \item Else return $\cm$ where $\cm = D_{\csubk{0}}(\csubc{0})$ 
            \end{enumerate}
            \label{fig:ind-cca-dec}
        \end{subfigure}
    }
    \caption{How To Make an IND-CCA Scheme}
    \label{fig:ind-cca-ed}
    \end{figure}

    This obviously involves the ciphertext being expanded (more than it might be already) because it includes a MAC. 

    With the MAC allowing us to validate a ciphertext as being `authentic', this type of scheme is often called \emph{`authenticated encryption'}.
    It is important to realise the \textbf{very important} distinction between this idea, which is that a ciphertext is authentic, and the concept that a ciphertext came from where it was supposed to, which is part of the regular cryptographic definition of `Authenticity'.

    \subsection{How To Make A MAC}
    There are a few types of MAC schemes, from MACs that are custom made for that specific encryption scheme to MACs that are derived from MDCs (Manipulation Detection Codes).

    The most popular, and exceedingly examinable, variants that we look at are all based on CBC-mode block ciphers. They are covered by various international standards\footnote{Gotta love standards. Thrilling stuff} and are very widely used. We shall refer to schemes like this under the title of CBC-MAC. Like this...

    \subsection{CBC-MAC}
    CBC-MAC is pretty much a straightforward application of a block cipher (like DES or AES). We just have to add the MAC on after padding the ciphertext.\\
    \\
    Given our cipher, which operates on blocks of \emph{b} bits, we can make a MAC really simply; assuming we have \emph{q} data blocks in our message, $\csubm{1}, \csubm{2}...m_{q}$, we would do as follows:

    \begin{enumerate}
        \item Set initial intermediate variables: $I_{1} = \csubm{1}, O_{1} = e_{\ck}(I_{1})$
        \item For $i = 2, 3 ... q$:
            \begin{itemize}
                \item $I_{i} = \csubm{i} \oplus O_{i - 1}$
                \item $O_{i} = e_{\ck}(I_{i})$
            \end{itemize}
        \item Do any post-processing you want on $O_{q}$.
        \item Truncate if necessary to $m$ bits and serve piping hot.
    \end{enumerate}
    In terms of padding the ciphertext, there are 3 schemes suggested in those groovy groovy standards. All of these pad to a whole number of blocks.
    \begin{itemize}
        \item Just add zeroes. Easy, right?
        \item Add a single 1, then trail out with zeroes.
        \item Pad with zeroes until you get a whole number of blocks, then add a block containing the length of the unpadded message.
    \end{itemize}
    In terms of optional post-processing, there are two specified methods:
    \begin{itemize}
        \item Choose another key, $\csubk{1}$, and replace $O_{q}$ with $e_{\ck}(d_{\csubk{1}}(O_{q}))$
        \item Choose another key, $\csubk{1}$, and replace $O_{q}$ with $e_{\ck_{1}}(O_{\cq})$ 
    \end{itemize}
    Either one of those can make it much harder to do a brute force search for $k$. Which is a good thing, young padawan. Oh yes.

    \subsection{Hashing}
    Like we said, hashing in this case is just an efficient function mapping arbritrarily long binary strings to fixed length binary strings. Simples.\\
    \\
    Incidentally, these hashes are essentially MCDs, though they are not called that now. Used to be the case that you'd do a hash (MDC), concatenate it to the message and then encrypt that. The problem with a concatenate-then-encrypt scheme under CBC is that an attacker can create a message that consists of the message they wish to send, a hash for it, and then some other random guff, and then hash that (this is a CPA method). If the attack then truncates the resulting ciphertext they get their valid target ciphertext, complete with hash.\\
    % TODO: EXPLAIN THIS BETTER DRUMMOND YOU CRAZY MAN.
    \\
    While earlier we talked about hashing with a key, in practice hash functions don't have a key. It is often simpler to consider a family of functions, and require three conditions of them:
    \begin{enumerate}
        \item Preimage Resistance: given $\cc = h(\cm)$, hard to find another $\cm^{'}$ such that $h(\cm^{'}) = \cc = h(\cm)$
        \item 2nd Preimage Resistance: The same as preimage resistance, except that we are also given $\cm$.
        \item Collision Resistance: hard to find $\cm, \cm^{'} \neq \cm$ such that $h(\cm) = h(\cm^{'})$
    \end{enumerate}
    Typical practical choices for a cryptographic hash are the SHA family, or the RIPEMD family. % What the fuck is a RIPEMD? Sounds bad lol. - Matt

    \subsection{From Hash To MAC}

    Given a hash function, how can we bung a key in there sensibly? There are the simple prefix, suffix and envelope methods, though these are not secure.
    \begin{itemize}
        \item With prefixing it is possible to calculate $MAC_{\ck}(\cm||\cm^{'})$ without knowing the key. Simply split your intended message in half, and you're good to go. Exact method not shown in notes.
        \item Suffixing suffers from a weakness that makes it easier to find collisions in the hash function; this can be done offline, as well, so you do not need to send lots of queries to the target. Exact methodology not disclosed in notes.
        \item Envelope method with padding: The reason this isn't secure isn't given in the notes, and I can't find it online. % TODO: SOMEONE HELP ME OH GOD PLEASE
    \end{itemize}
    It is better to use HMAC (keyed-\textbf{H}ash \textbf{M}essage \textbf{A}uthentication \textbf{C}ode):
    $$HMAC_{\ck}(\cm) = h(\ck||p_{1}||h(\ck||p_{2}||\cm))$$
    Where $p_{1}$ and $p_{2}$ are fixed strings used to pad $\ck$ to a full block.\\
    \\
    You can use both MDC and MAC for data integrity, with or without confidentiality.\\
    \\
    Without confidentiality:
    \begin{itemize}
        \item \textbf{MAC}: compute $MAC_{\ck}(\cm)$ and send $\cm||MAC_{\ck}(\cm)$
        \item \textbf{MDC}: Send $h(\cm)$ over a seperate, authenticated channel.
    \end{itemize}
    With confidentiality:
    \begin{itemize}
        \item \textbf{MAC}: need two different keys, $k_{1}$ and $k_{2}$
        \begin{itemize}
            \item $\csubk{1}$ is for computing $\cc = e_{\csubk{1}}(\cm)$
            \item $\csubk{2}$ if for computing $MAC_{\csubk{2}}(\cc)$, which you append to $\cc$ before sending that combined string.
        \end{itemize}
        \item \textbf{MDC}: we only need one key $k$ for encryption, where we send $\cc = e_{\ck}(\cm||h(\cm))$, but as we discussed earlier this can be compromised by a man in the middle attack.
    \end{itemize}


\chaptersub{Asymmetric Modes}{Public Key Cryptography}

\section{Introduction}

	The basic concept of public key cryptography is the idea of a box with two keys, a public key, and a private key. The \textbf{public key} allows anyone to \textbf{leave a message}. They cannot however read each others messages. The \textbf{private key} by contrast, allows its owner to \textbf{read any message left}. It is this asymmetry that gives the scheme its name.\\
	\\
	Or to put	it more concretely:\\
	\\
	\textbf{Message + \textcolor{B}{Public Key} = Ciphertext}\\
	\textbf{Ciphertext + \textcolor{R}{Private Key} = Message}\\

	Throughout this chapter, you will see a lot of colours. If you see \textcolor{R}{red}, the highlighted symbols are relevant to the \textcolor{R}{private key}. If you see \textcolor{B}{blue}, the highlighted symbols are relevant to the \textcolor{B}{public key}.

\section{Vanilla RSA}

	\subsection{Definition}

		RSA is actually very simple. Lets start with the encryption and decryption definitions, and work back from there.

		$$ c = m^{\textcolor{B}{e}} \bmod \textcolor{B}{N} $$

		$$ m = c^{\textcolor{R}{d}} \bmod \textcolor{B}{N} $$
		Simple right? $e$, $d$, and $N$ are the only things we need. So now lets cover just what these are.\\
		\\
	  \begin{tabularx}{\linewidth}{l l X}
	  \textbf{Symbol} & \textbf{Maths}\footnote{Any unfamiliar symbols or functions are described in the Mathematics section.} & \textbf{Notes}\\
	  $\textcolor{B}{N}$ & $\textcolor{B}{N} = \textcolor{R}{p} * \textcolor{R}{q}$ & Where $p$ and $q$ are two massive \textbf{prime} numbers.
	  \\
	  $\textcolor{B}{e}$ & $\gcd(\textcolor{B}{e},\phi(\textcolor{B}{N}))=1$ & A randomly chosen integer, where $1 < \textcolor{B}{e} < \phi(\textcolor{B}{N})$.
	  \\
	  $\textcolor{R}{d}$ & $\textcolor{B}{e}*\textcolor{R}{d} = 1 \mod \phi(\textcolor{B}{N})$ & Computed using the XGCD algorithm.
	  \\
	  \end{tabularx}
	  
	It is worth proving that this system works, namely that decrpyting a ciphertext will always result in the original message.

	\vspace{5mm}
	
	Let $c = enc(m)$, and let $x = dec(c)$. We intend to prove that $x \cong m \mod N$. The system is obviously correct (and obviously insecure) in the case where $c=m=0$, so suppose this is not the case. We have that
	
	\begin{align*}
	    x &\cong m^{\blue{e}\red{d}} \mod \blue{N}\\
	    x &\cong m^{\blue{e}\red{d}} \mod \red{p}\\
	    x &\cong m^{\blue{e}\red{d}} \mod \red{q}
	\end{align*}
	
	Consider $x \cong m^{\blue{e}\red{d}} \mod \red{p}$. Since $m \neq 0$, we can invoke Fermat's Little Theorem to conclude that
	
	$$
	    m^{\blue{e}\red{d}} \cong m^{\blue{e}\red{d} \mod (\red{p}-1)} \mod \red{p}
	$$
	
	Since $\blue{e}\red{d} \cong 1 \mod (\red{p}-1)(\red{q}-1)$, we have that, for some integer $k$, $\blue{e}\red{d} = 1 + k(\red{p}-1)(\red{q}-1)$, hence $\blue{e}\red{d} \cong 1 \mod (\red{p}-1)$. Therefore
	
	$$
	    m^{\blue{e}\red{d} \mod (\red{p}-1)}\cong m^1 \mod \red{p}
	$$
	
	So $x \cong m \mod \red{p}$. Nearly identical reasoning allows us to conclude that $x \cong m \mod \red{q}$. By the Chinese Remainder Theorem, it must therefore be the case that $x \cong m \mod N$, and hence
	
	$$
	    dec(enc(m)) \cong m \mod N 
	$$
	
	\begin{flushright}
	QED
	\end{flushright}
	\subsection{Security}

		Vanilla RSA is \textbf{OW-CPA} under the assumption that the RSA problem is hard. If we could easily break vanilla RSA with a \textbf{OW-CPA}, then we could use this attack to easily solve the RSA problem. Since we assumed that we cannot easily solve the RSA problem, it must be the case that we cannot mount a \textbf{OW-CPA} attack against vanilla RSA.\\
		\\
		It is however \textbf{not IND-CPA} secure. This is pretty obvious; the encryption function is deterministic, so when given a ciphertext from a set of two messages, we can simply encrypt one message from the set, and know that if it does not match the ciphertext we received, we must have been sent the other message.\\
		\\
		An encryption scheme is considered \textbf{malleable} if given $c_1$, the ciphertext of $m_1$, we can compute another valid ciphertext $c'$, from a message mathematically related to $m_1$. This is indeed the case with RSA, for instance given two ciphertexts $c_1$ and $c_2$, we could compute:
		$$c_3 = c_1*c_2 \mod N = (m_1 * m_2)^e \mod N$$ 
		Without ever knowing $m_1$ or $m_2$! This is not a good thing.\footnote{Unless you do Applied Security, in which case this was what you used to perform the RSA-OAEP attack successfully.} Why not? Because if you encrypt the number 100, without knowing that, or the private key, I can create a valid ciphertext for the number 200, simply by encrypting $m = 2$ and multiplying our ciphertexts together, modulo $N$.\\
		\\
		Due to this malleability, vanilla RSA is \textbf{not OW-CCA} secure. Recall in a CCA we are allowed to decrypt any ciphertext \textit{besides the target ciphertext}. However, we now know how to manipulate the ciphertext while leaving the original message relatively unharmed. As such, to recover $m$, we simply compute $c' = c*2^e \mod N$, then decrypt $c'$, giving $m' = m * 2$. Recovering $m$ is now trivial.


\section{Signatures}



\section{Hybrid Encryption}

\section{Key Encapsulation}

\section{Padding Schemes}

	\subsection{Introduction}
		A padding scheme in its simplest form is a system to ensure that when our block size does not exactly divide our message, we do not lose data, or gain data that was not in our message (i.e. we are able to distinguish padding from our original message body).\\
		\\
		A padding scheme generally uses a block at either the beginning or start of the message to specify the length of the message body, potentially along with additional information, such as a hash to verify message authenticity.

	\subsection{OAEP}
		Optical Asymmetric Encryption Padding is one such padding scheme. \textit{Incredibly simplified}\footnote{Seriously this is so simplified it is only for ones intuitive understanding of why it is effective} the scheme pads a message with a --- generally SHA1 --- hash of the message body. If after decryption the hash is not correct for this message body, we know the message is invalid. OAEP goes above and beyond this, but that's the gist of it.\\
		\\
		When used with RSA, OAEP gives a scheme which is \textbf{IND-CCA} secure.


\section{Rabin}
	Rabin encryption is a public key cryptography scheme that is provably more secure than RSA. Its security is based on the difficulty of the SQROOT problem, which is provably equal in difficulty to the FACTORING problem. By contrast, the RSA problem, while assumed to be hard, has not been proven to be hard.\\
	\\
	Our private key is made up of two components, $\textcolor{R}{p}$ and $\textcolor{R}{q}$, two similarly sized prime numbers where $\textcolor{R}{p} = \textcolor{R}{q} = 3 \mod 4$.\footnote{This just makes extracting roots fast, there is no cryptographic need for this.}\\
	\\
	Our public key is made up of two components as well, $\textcolor{B}{N} = \textcolor{R}{p}*\textcolor{R}{q}$ and $\textcolor{B}{B}$. $\textcolor{B}{B}$ is a randomly chosen number between $0$ and $\textcolor{B}{N}$\\
  \begin{figure}[htp!]
		$$c = m * (m + \textcolor{B}{B}) \mod \textcolor{B}{N}$$
  \caption{Rabin Encryption Algorithm}
  \label{fig:rabin-enc}
  \end{figure}

  \begin{figure}[htp!]
		$$m = \sqrt{\frac{\textcolor{B}{B}^2}{4}+c} - \frac{\textcolor{B}{B}}{2} \mod \textcolor{B}{N}$$
  \caption{Rabin Decryption Algorithm}
  \label{fig:rabin-dec}
  \end{figure}
	An interesting property of the Rabin scheme, as seen in Figure~\ref{fig:rabin-dec}, the private key is not strictly speaking needed for decryption, however in reality this is a case of the SQROOT problem; even knowing the contents of the square root, finding the square root is not a trivial task.\\
	\\
	This is where the private key comes in. Lets focus on the portion of the equation that is hard to solve:
	$$t = \frac{\textcolor{B}{B}^2}{4}+c$$
	We instead now solve $\sqrt{t} = \pm x\mod p $ and $\sqrt{t} = \pm y \mod q$. The final step is to make use of the CRT to solve $\mod N$. % TODO: I'm stopping here because at this point I'm lost. - Matt


\section{ElGamal}









































\part{Cryptography B}

\chapter{Lecture 1}%1111111111111111111111111111111111111111111111111111111111111111111111

These brief lecture notes are intended to help you focus on the main concepts that we have covered in class. Their structure follows closely on that of the lectures. These notes are not a substitute for your own - they are not as comprehensive as they should be and may contain typos. Apart from these notes you should also read the respective chapters in Katz-Lindell (KL).

\section{Notation}

We write \begin{math}x \gets A(y)\end{math} to say that we obtain \textit{x} by running algorithm A on input \textit{y}. Notice that \textit{A} can be randomized, which is indicated as follows: \begin{math}x\xleftarrow{\$}A(y)\end{math}. In general if \textit{D} is a distribution we write \begin{math}x\xleftarrow{\$}D\end{math} to indicate that \textit{x} is sampled according to ditribution D. If \textit{S} is any finite set, we write \begin{math}x\gets S\end{math} to mean that we select \textit{x} uniformly at random from \textit{S}. If \textit{x} and \textit{y} are are strings, then we write \begin{math}x||y\end{math} for the concatenation of \textit{x} with \textit{y}.

\section{Security Models}


In the early days of cryptography, the design of cryptographic systems used a trial-and error approach: first design the system, then wait for someone to break it, then patch it, wait for another break and so on, until no more attacks can be found. Unfortunately this approach offers minimal security guarantees.

The defining characteristic of modern cryptography is that it uses \textit{security models} for defining the security of various primitives and protocols and perhaps more importantly, their use in \textit{proving security}. Next, we describe at high level what a security models is and then have a look at some examples.

Before being able to even define a security model, one has to fix(or set) the syntax of a primitive, that is, to specify precisely what are the algorithms that implement the primitive. A security model for a primitive then makes two things precise:

\begin{itemize}
\item The first thing that a security model for a primitive specifies is how an arbitrary adversary is allowed to interact with the primitive. This interaction should be as general as possible since it should reflect \textit{all} possible ways in which the primitive is going to be used when deployed.
\item The second issues thst the security model should specify is wehat is the adversary`s goal, in other words, what should be considered a break for the primitive.
\end{itemize}

We will describe several of the most common security models used in modern cryptography. How do we know that a secuurity model does indeed capture the security of the primitive? Unfortunately there is no good answer to this question. It may seem that security models suffer from the same problem as the first cryptosustems, however, in this case the situation is somewhat better: once validated, through inspection, usage, etc. they can be used to \textit{prove} the security of particular implementations. It still happens in cryptography that the models in use turn out not to be sufficiently precise because they do not capture all possible abilities that an adversary has or because it does not specify what a break is sufficiently well.

\section{Negligible Functions}

A function \textit{f}, is said to be negligible if it decreases faster than the inverse of any polynomial. Or, more formally, for any polynomial \textit{p} there exists some natural number \begin{math}n_0 \in \mathbb{N} \end{math} such that for all \begin{math} n\geq n_0\end{math} we have that:
\begin{align*}f(n) \leq \frac{1}{p(n)} \end{align*}

\section{Exercises}
\begin{enumerate}
\item Prove that if \textit{f(x)} is negligible then $c\times f(x)$ is also negligible, where c is a constant.
\item Prove that if \textit{f(x)} and \textit{g(x)} are negligible then \textit{f(x)+g(x)} is negligible too.
\end{enumerate}

\chapter{Lecture 2} %222222222222222222222222222222222222222222222222222222222222222

\section{One-Way-Functions(OWF)}

We were looking at the diagram in lecture one and asked the question: What would happen if RSA or Discrete Log are broken? What will we rely on then? The answer is that there are other hard problems out there for example: eliptic curves, factoring, lattice problems.

What is hard on those problems is distilled into one single concept: the One Way Function. What is great about this type of functions is that they are easy to compute but hard to invert (which means encryptions based on them are easy to compute but hard to break).

But how do we mathematically define the fact that this function is "easy to compute"? We say that a function is easy to compute if there is a poly-time algorithm that computes that function.

Now how do we define that it's hard to invert? We can informally state that as the probability of an adversary to find a pre-image for a given image of a random input is negligible.

The two properties above give us the definition of an OWF which is formally stated as follows:

A function \begin{math}f:\{0,1\}^*\leftarrow\{0,1\}^*\end{math} is an OWF if:
\begin{itemize}
\item \begin{math}\exists  M \text{, a poly time algorithm which computes } f\end{math}
\item    for any efficient adversary \textit{A},\\ \begin{math} Prob[y=f(z): x \xleftarrow{\$} \{0,1\}^n; y=f(x);z \xleftarrow{\$} A(y)] \text{
    \hspace{2mm}is a negligible function of }k\end{math}
\end{itemize}

Now let's try a couple of exercises:
\begin{itemize}

    \item Is the function \begin{math}f(x)=0\end{math} an OWF?
    
    \begin{itemize}
    
        \item Clearly this function is not reversible since we cannot recover \textit{x} given \textit{f(x)}. However,               given the image of \textit{f} on a random point we can easily find a pre-image by simply picking any $z \in \{0,1\}^*$  proving that this function is not an OWF.
        
    \end{itemize}
    
    \item Is the function \begin{math}f(x_1,x_2)=x_1\end{math} an OWF? Consider \begin{math}x=x_1||x_2\text{, } |x_1|=|x_2|\end{math}.
    
    \begin{itemize}
    
        \item Clearly this function is not reversible in the traditional sense, however, given $y=x_1=f(x_1,x_2)$ one can easily find a pre-image of $y$ by simply considering $x_1,0^{|x_1|}$ (where by $0^n$ we mean $0$ repeated $n$ times).
        
    \end{itemize}

    \item Is the function \begin{math}f(x_1,x_2)=x_1\oplus x_2\end{math} an OWF? This is ``One-Time Pad``.
    
    \begin{itemize}
    
        \item This function is not reversible either, however, given $y=x_1=f(x_1,x_2)$ one can easily find a pre-image of $y$ by simply considering $(x_1\oplus x_2)$ and $0^{|x_1\oplus x_2|}$ (where by $0^n$ we mean $0$ repeated $n$ times).
        
    \end{itemize}

    \item Is the function \begin{math}f(x_1,x_2)=x_1\times x_2\end{math} an OWF?
    
    \begin{itemize}
    
        \item The definition says that in order for a function to be a OWF, the probability of an adversary to find a pre-image must be a negligible function. In this case, in contrast to the ones above, the probability of finding a pre-image is not 1 anymore. This problem is based on the factoring problem. The probability of finding a factor is at least $\frac {1}{2}$ because of the number 2. However, if we choose our numbers to be large primes then this probability lowers considerably and it can get to the point when it is negligible, rendering the function \begin{math}f(x_1,x_2)=x_1\times x_2\end{math} an OWF.
        
    \end{itemize}
    
\end{itemize}
    
From the last example we conclude that \textit{``if Factorising is HARD then OWF is HARD``}.

\section{One-Time Signatures(OTS)}

Using OWFs we can build \textit{signatures} and \textit{symmetric encryption schemes}, but we cannot build \textit{public-key encryption}.\\

\textit{\textbf{Definition:} A signature scheme is a tuple of three probabilistic polynomial-time algorithms \textbf{(Gen, Sign, Verify)} satisfying the following:}
\begin{enumerate}
    \item The key-generation algorithm Gen takes as input a security parameter $1^n$ and outputs a pair of keys \textit{(pk,sk)}. These are called the public key(or the verification key - vk) and the private key, respectively. We assume for convenience that \textit{pk} and \textit{sk} each have length at least \textit{n} and that \textit{n} can be determined from \textit{pk,sk}.
    \item The signing algorithm \textit{Sign} takes as input the private key \textit{sk} and a message $m \in \{0,1\}^*$. It outputs a signature $\sigma$, denoted as $\sigma \gets Sign_{sk}(m)$.
    \item The deterministic verification algorithm \textit{Verify} takes as input a public key \textit{pk}, a message \textit{m} and a signature $\sigma$ and outputs a bit \textit{b}, with \textit{b} = 1 for valid and \textit{b}=0 for invalid. We write this as $b:=Verify(m,\sigma)$.
\end{enumerate}

$Verify(m, Sign(m)) = 1$ - the message m is authentic.\\

Now we shall construct a \textit{One-Time Signature} using OWFs.\\

\textit{\textbf{Theorem:} If $\exists \text{OWF} \text{ then } \exists \text{OTSs(One-time Signatures)}$}.
\newpage
In order to define the security of a signature scheme we build a game:
\begin{itemize}
\item The key generation algorithm outputs a public key (pk) and a private key (sk).
\item The adversary A gets the public key (pk)
\item The signing oracle receives the private key(sk) from Gen and the message m
\item The adversary A sends a message m to the signing oracle.
\item The signing oracle outputs a signature $\sigma$
\end{itemize}

What does it mean for A to produce a forgery?

\textbf{Definition} A forgery in a signature scheme has been produced if an adversary manages to find a message $m^*$ different from the original $m$ and a signature $\sigma^*$ such that $Verify(pk, m^*, \sigma^*)=1$.\\

The scheme is secure if no adversary can produce any forgery.\\

An OTS a signature scheme that is built using any OWF and assumes that the adversary $A$ can request a signature from the signing oracle ONLY ONCE.\\

\textbf{Theorem:} If $\Pi$ is EUF-CMA(Existential Unforgeability under Chosen Message Attack) then $\Pi$ is a secure OTS scheme.\\

\textbf{The Lamport Signature Scheme}\\\\
$f$ is an OWF.\\
for i$\gets$ 1 to l do\\\\
$x_{i0} \in \{0,1\}^n$\\
$x_{i1} \in \{0,1\}^n$\\\\
$y_{i0} \gets f(x_i,0)$\\
$y_{i1} \gets f(x_i,1)$\\\\
$s_k = \begin{pmatrix}
  x_{1,0} & x_{2,0} & \cdots & x_{l,0} \\
  x_{1,1} & x_{2,1} & \cdots & x_{l,1} \\
\end{pmatrix}$\\\\
$v_k = \begin{pmatrix}
  y_{1,0} & y_{2,0} & \cdots & y_{l,0} \\
  y_{1,1} & y_{2,1} & \cdots & y_{l,1} \\
\end{pmatrix}$\\\\
$Sign\begin{pmatrix}
  s_k, & m_1 & m_2 & \cdots & m_l \\
\end{pmatrix}=\begin{pmatrix}
  x_{1 m_1} & x_{2 m_2} & \cdots & x_{l m_l} \\
\end{pmatrix}$\\\\
$Verify\begin{pmatrix}
  v_k, & m_1 & m_2 & \cdots & m_l, & x_1 &x_2 & \cdots & x_l\\
\end{pmatrix}=\bigwedge^{n}_{i=1}f(x_i) == y_{i m_i}$\\

Breaking this signature would require for the adversary to invert the OWF.\\

However if the adversary can access the sign oracle more than once he would be able to breake it.

\chapter{Lecture 3}% 333333333333333333333333333333333333333333333333333333333333333333
\section{OTS - Proof of Security}

An OTS is unforgable if:\\
$Pr[Sig-forge^{1-time}_{A,\Pi}(n) = 1] \leq negl(n).$\\\\

\textbf{Theorem:} Let l be any polynomial. If f is an OWF, then $\Pi$ is a secure OTS scheme.

\textbf{Proof:}To prove this we will use reduction. We suppose that OTS is not secure, which means it can be broken by an adversary A that has the power to return one x (knowing the pk) in one point not known by the adversary B. We then build an adversary B who will use adversary A's break to invert OWF.

To help us understand the proof for the general case we will first go through two particular cases where we make the adversary A stronger.\\\\

Let $f(q)=w$ be a OWF.\\
Adversary A can find the $x_{i,b}$ for a given $y{i,b}$ with a \textbf{non-negligible} probability $\varepsilon$.
Adversary B knows $w$ and wants to find out $q$. B has access to A so all it has to do is to replace one of the original $ys$ with his $w$ and send the new $pk$ to A. A then returns $q$ which is the inverse of $w$, therefore inverting $f$.\\

\begin{enumerate}
\item \textbf{Adversary A can get all the $x_{i,0}$ given the public key.\\\\}
Adversary A knowing  $\begin{pmatrix}
  y_{1,0} & y_{2,0} & \cdots & y_{l,0} \\
  y_{1,1} & y_{2,1} & \cdots & y_{l,1} \\
\end{pmatrix}$ can find $\begin{pmatrix}
  x_{1,0} & x_{2,0} & \cdots & x_{l,0} \\
  . & . & \cdots & . \\
\end{pmatrix}$.\\\\\\
B sends $\begin{pmatrix}
  w & y_{2,0} & \cdots & y_{l,0} \\
  y_{1,1} & y_{2,1} & \cdots & y_{l,1} \\
\end{pmatrix}$ to A.\\\\
A returns $\begin{pmatrix}
  q & x_{2,0} & \cdots & x_{l,0} \\
  . & . & \cdots & . \\
\end{pmatrix}$ to B.\\\\\\
Therefore the probability of B inverting the OWF $f$ is $\varepsilon$.\\\\\\
\item \textbf{Adversary A returns randomly either $x_{1,0}$ or $x_{1,1}$. }\\
Adversary A knowing  $\begin{pmatrix}
  y_{1,0} & y_{2,0} & \cdots & y_{l,0} \\
  y_{1,1} & y_{2,1} & \cdots & y_{l,1} \\
\end{pmatrix}$ returns randomly one of the following: \\\\\\ $\begin{pmatrix}
  x_{1,0} & . & \cdots & . \\
  . & . & \cdots & . \\
\end{pmatrix}$ or $\begin{pmatrix}
  . & . & \cdots & . \\
  x_{1,1} & . & \cdots & . \\
\end{pmatrix}$.\\\\\\
B flips a coin and randomly sends to A one of the following:\\\\$\begin{pmatrix}
  w & y_{2,0} & \cdots & y_{l,0} \\
  y_{1,1} & y_{2,1} & \cdots & y_{l,1} \\
\end{pmatrix}$ or $\begin{pmatrix}
  y_{1,0}& y_{2,0} & \cdots & y_{l,0} \\
  w & y_{2,1} & \cdots & y_{l,1} \\
\end{pmatrix}$.\\\\\\
A randomly returns to B one of the following:\\\\$\begin{pmatrix}
  r & . & \cdots & . \\
  . & . & \cdots & . \\
\end{pmatrix}$ or $\begin{pmatrix}
  . & . & \cdots & . \\
  r & . & \cdots & . \\
\end{pmatrix}$.\\\\\\
The probability that $r=q$ and that B inverts $f$ is $\frac{\varepsilon}{2}$.
\end{enumerate}

\textbf{Now, for the general case,  we assume that adversary A can return randomly between $x_{i,0}$ or $x_{i,1}$, where $i$ is a random value.}\\

Adversary A knowing  $\begin{pmatrix}
  y_{1,0} & y_{2,0} & \cdots & y_{l,0} \\
  y_{1,1} & y_{2,1} & \cdots & y_{l,1} \\
\end{pmatrix}$ returns randomly one of the following:\\\\\\$\begin{pmatrix}
  . & . & \cdots & x_{i,0}& \cdots & . \\
  . & . & \cdots & . & \cdots & . \\
\end{pmatrix}$ or $\begin{pmatrix}
  . & . & \cdots & . & \cdots & . \\
  . & . & \cdots & x_{i,1}& \cdots & . \\
\end{pmatrix}$where $i$ is a random position.\\\\

Adversary B chooses a random position $j$ from $1$ to $l$, then flips a coin and randomly sends one of the following:\\\\ $\begin{pmatrix}
  y_{1,0} & y_{2,0} & \cdots w_j& \cdots& y_{l,0} \\
  y_{1,1} & y_{2,1} & \cdots y_{j,1}& \cdots& y_{l,1} \\
\end{pmatrix}$ or  $\begin{pmatrix}
  y_{1,0} & y_{2,0} & \cdots & y_{j,0} & \cdots& y_{l,0} \\
  y_{1,1} & y_{2,1} & \cdots & w_j & \cdots& y_{l,1} \\
\end{pmatrix}$.\\\\\\
The adversary A returns randomly one of the following:\\\\$\begin{pmatrix}
  . & . & \cdots & . & \cdots & . \\
  . & . & \cdots & r_{i,1}& \cdots & . \\
\end{pmatrix}$ or $\begin{pmatrix}
  . & . & \cdots & r_{i,0} & \cdots & . \\
  . & . & \cdots & .& \cdots & . \\
\end{pmatrix}$\\\\\\

The probability that $j$ is the same position as $i$ is $\frac{1}{l}$. The probability that \newline $(r=q) \wedge (i=j)$ is $\frac{1}{2l}$, Therefore the probability thet B inverts the OWF $f$ is $\frac{\varepsilon}{2l}$.\\\\
\textbf{Algorithm I:}\\
The algorithm is given $y$ and $1^n$ as input.
\begin{enumerate}
\item Choose random $i^*\gets\{1,\ldots,l\} \text{ and } b^*\gets\{0,1\}$. Set $y_{i^*,b^*} := y.$
\item For all $i\in\{1,\ldots, l\}$ and $b\in\{0,1\}$ with $(i,b)\not=(i^*,b^*):$ 
\begin{itemize}
\item Choose $x_{i,b}\gets\{0,1\}^n \text{ and set } y_{i,b}:=f(x_{i,b}).$
\end{itemize}
\item Run $A$ on input $pk:= \begin{pmatrix}
  y_{1,0} & y_{2,0} & \cdots & y_{l,0} \\
  y_{1,1} & y_{2,1} & \cdots & y_{l,1} \\
\end{pmatrix}$.
\item When $A$ requests a signature on the message $m'$:
\begin{itemize}
\item If $m'_{i^*}=b^*$, stop.
\item Otherwise, return the correct signature $\sigma = (x_{1,m'_1},\ldots,x_{l,m'_l}).$
\end{itemize}
\item When $A$ outputs $(m,\sigma)$ with $\sigma = (x_1,\ldots,x_p):$
\begin{itemize}
\item If $A$ outputs a forgery at $i^*, b^*)$, then output $x_{i^*}$.
\end{itemize}
\end{enumerate}

\section{Hard-Core Bits (HCB)}
Note: Hard-core bits are also called hard-core predicates.\\\\
By definition a OWF is hard to invert but that does not mean that we cannot learn \textbf{any} information about $f(x)$.\\\\
Let's consider a trivial example where $f(x)$ is an OWF.\\
Consider $g(x_1, x_2) = (x_1,f(x_2)) \text{ where } |x_1|=|x_2|$. It can be easily proven that $g$ is also an OWF even though it \textbf{reveals half of the input}.\\\\
The informal definition of a HCB:\\
\textbf{Definition 1:} $B\{0,1\}^n \rightarrow \{ 0,1 \}$\\
$B$ is a HCB for a function $f$ if given $f(x)$ you cannot figure out what $B(x)$ is.\\\\
The formal definition of a HCB:\\
\textbf{Definition 2:} A function $B\{0,1\}^n\rightarrow\{0,1\}$ is a HCB for a function $f$ if:
\begin{enumerate}
\item $B$ can be computed in polynomial time
\item for every probabilistic polynomial-time algorithm $A$ there exists a negligible function $negl$ such that:\\
$Prob[A(f(x))=B(x), x\gets\{0,1\}^n]\leq \frac{1}{2} + negl(n)$\\\\
Consider:
\begin{itemize}
\item $B(x_1,\ldots,x_n)=x_1 \oplus x_2 \oplus \ldots \oplus x_n$.\\
This example intuitively looks like a HCB. Because $f$ is an OWF and at least one of the bits should be hidden, you would think that computing $\bigoplus^n_{i=1}x_i$ would be impossible without inverting $f$. 
That is not true and we can prove it by defining $g(x)=(f(x), \bigoplus^n_{i=1}x_i)$ which is an OWF and clearly does not hide $B(x) = \bigoplus^n_{i=1}x_i$.\\ 
\end{itemize}
\end{enumerate}
\section{Goldreich and Levin Proof}

\textbf{Theorem:} If $f$ is an OWF then $\exists$ an OWF $gl$ and a HCB for $g$. If $f$ is a permutation the so is $g$.

$f:\{0,1\}^*\leftarrow\{0,1\}^*$\\\\
Define $g(x,R) = (f(x), R) \text{ where } |x| =|R|$\\
$B(x,R)= <x,R> =\bigoplus^n_{i=1}x_i \times R_i$\\
$B(x,R)$ is a HCB for $g$.\\\\
Example:
\begin{itemize}
\item $<1011,0110> = (1\times0)\oplus(0\times1)\oplus(1\times1)\oplus(1\times0)=0\oplus0\oplus1\oplus0=1$.
\end{itemize}
\chapter{Lecture 4}%444444444444444444444444444444444444444444444444444444444444444444444444
\section{Indistinguishability}
Randomness is very important in cryptography so that you cannot tell when the same message has been encrypted by looking at the cyphertexts.\\

\textbf{Definition 1:} Given X and Y distributions, we say that X and Y are computationaly indistinguishable if no efficient distinguisher can tell the distributions apart.\\\\
$[Prob_{z\gets X}[1\gets D(z)]-Prob_{z\gets Y}[1\gets D(Z)]]\leq negl(something*)$\\\\
We will ignore the definition of the \textit{something*}. What we can say about it for now is that it is the length of the key.\\\\
\textbf{Notation:} $X\sim Y$ means X is computationaly indistinguishable from Y.\\

Another definition of indistinguishability can be expressed in terms of HCB`s:\\
\textbf{Definition 2:} $B$ is a HCB for $f$.\\
Let:
\begin{itemize}
\item $X=(f(x), B(x)) \text{ with } x\xleftarrow{\$}\{0,1\}^n$
\item $Y=(f(x), b) \text{ with } x\xleftarrow{\$}\{0,1\}^n \text{ and } b\xleftarrow{\$}\{0,1\}^n$
\end{itemize}
Then $X\sim Y$.

\section{Exercises:}
\begin{itemize}
\item Prove that the two definitions above are equivalent
\end{itemize}
\newpage
\section{Pseudo-Random Generators (PRG`s)}

$G:\{0,1\}^n\rightarrow\{0,1\}^{f(n)}$, where $l(n)$ is some polynomial.
$G$ is a PRG if:
\begin{enumerate}
\item$n<l(n)$
\item $G(s) \sim U_{l(n)}$ where $s\xleftarrow{\$}\{0,1\}^n$ and $U_t$ is an uniform distribution on a bitstring of length $t$
\end{enumerate}
\textbf{Theorem:} Let $f$ be an One Way Permutation (OWP) and let B be a HCB for $f$.
Then the algorithm $G(s)=(f(s),B(s))$ is a PRG with $l(n)=n+1$.\\\\
\textbf{Proof:}
\chapter{Lecture 5}%5555555555555555555555555555555555555555555555555555555555555555
\section{Pseudo-Random Functions(PRF`s)}

%\chapter{Lecture 6}%6666666666666666666666666666666666666666666666666666666666666666
%\chapter{Lecture 7}%7777777777777777777777777777777777777777777777777777777777777777
%\chapter{Lecture 8}%8888888888888888888888888888888888888888888888888888888888888888

















\part{Problems Class Model Answers}
\section{Cryptography A}
\subsection{Problems Class 1}
Example slides...

\subsection{Problems Class 2}

\end{document}

