
\chapter{Mathematics}
% Alternative title: kill me now. 
% I will buy the person that writes this section a Freddo. - Matt Bessey.
% Matt, you owe me a Freddo - Drum Ogilvie
    \section{Modular Arithmetic \& Groups}
    \subsection{Modular Arithmetic}
    Modular arithmetic is a bit odd at first, but you get used to it. It's all about remainders, really; we express modular relations in the following form:\\
    \begin{figure}
    \centering
        $$
        X \equiv \textit{Y \emph{\textbf{MOD}} Z}
        $$
    \end{figure}\\
    
    Where $X$ is any old number and $Z$ is the `Modulus'. Y is the remainder of the integer division of $X$ by $Z$, the result of an operation expressed in many languages (including C) as $X \% Z$. Simples.\\
    \\
    This leads us nicely onto groups.

    \subsection{Groups}
    A mathematical group is a combination of a set and an operation. So in our case, we use the set of remainders under some Modulus, defined thusly:
    \begin{figure}
    $$
        (\mathbb{Z}/N\mathbb{Z}) = \{0,..., N - 2, N - 1 \}
    $$

    \section{Multiplicative Inverse}
    The `inverse' for a group member is another member of the same group that when used as the second argument for the operation gives the result of the identity for that operation.\\
    \\
    In this case, the operation is multiplication, the multiplicative identity is 1, and our group is usually numbers Modulo N. So the inverse of one number Mod N is another number less than N, that when you multiply them together you get a number which is 1 Mod N. Still here? Still awake? Stay with me, soldier.
    
    \section{Greatest Common Divisor}
    Say you've got two numbers. What is the biggest factor that they both share? BOOM. 
    
    \section{Euler Phi Function}
    Also known as Euler's Totient Function
        
    \section{Lagrange's Theorem}
    
        \subsection{Fermat's Little Theorem}
        
    \section{Fields}
    
    \section{Chinese Remainder Theorem}
    
    \section{One Way Functions}
    
    %%%%%%%%%%%%%%%%%%%%%%%%%%%%%%%%%%%%%%%%
    % There's actually a few more sections %
    %%%%%%%%%%%%%%%%%%%%%%%%%%%%%%%%%%%%%%%%