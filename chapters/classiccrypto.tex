\chapter{Classical Cryptography}

\section{Perfectly Secret Encryption}
	\textbf{Definition:} An encryption scheme $(Gen,Enc,Dec)$ over a message space $\cm$ is perfectly secret if for every probability distribution over $\cm$, every message $\cm \in M$, and every ciphertext $\cc \in C$ for which $P(C=\cc)>0:$

	$$P(M=\cm|C=\cc)=P(M=\cm)$$

	In other words a scheme is perfectly secret if the distributions over messages and ciphertexts are independent.

\section{One Time Pad}
	The one time pad is a good system to review because it is \textbf{totally secure} / perfectly secret, if used correctly. Sadly, it is difficult to use in the real-world.\\
	\\
	It works by XOR-ing the entire message by the cipher. This means that the cipher must be as long as the message, which is clearly a limitation in the real world --- encrypting your hard drive would require another hard drive of data for the cipher.\footnote{The XOR operations outputs a 0 bit at every position in which both operands have the same bit, and a 1 in positions where they differ.}\\
	\\
	The other limitation is the fact it can only be used once. This is because an adversary  can ask you to encrypt a message ($\cc=\cm\oplus k$). They can then use that to calculate the key, $\ck$,  because they have both $\cc$ and $\cm$ ($\ck=\cm\oplus \cc$). In practice, this means that if you know part of a message being sent, you can use that to calculate the bits of cipher in those positions.

\section{Shannon's Theorem}
	
	Shannon's theorem provides a generalised definition for deeming whether a cryptosystem is \textbf{perfectly secret}. First, we assume we have a system with equal numbers of keys, plaintext, and ciphertexts; $\mathbf{\#K = \#P = \#C}$.\\
	\\
	We say that this system provides perfect secrecy iff:
	\begin{itemize}
		\item The probability of any key being used is $1/\#K$.
		\item For each $m \in P$ there exists a $c \in C$.
	\end{itemize}
	It's worth briefly looking into the \textit{why} of Shannon's theorem. If $\#K = \#P = \#C$, then given a key, there must be one mapping of each plaintext to each ciphertext. And if we try every key, then a given message must map to every ciphertext. If it does not, then we have leaked information about the message, for example if $m_1$ maps to $c_{666}$ more than once when encrypted with every key, then given $c_{666}$ we can say of all messages, it is more likely to be $m_1$ than another message, despite not even knowing the key!